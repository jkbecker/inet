\chapter{Modeling Power Consumption}
\label{cha:power}

\section{Overview}

Modeling power consumption becomes more and more important with the increasing
number of embedded devices and the upcoming Internet of things. Mobile personal
medical devices, large scale wireless environment monitoring devices, electric
vehicles, solar panels, low-power wireless sensors, etc. require paying special
attention to power consumption. The high fidelity simulation of power
consumption allows designing power sensitive routing protocols, MAC protocols,
physical layers, etc. which in turn results in more energy efficient devices.

In order to help the modeling process the power model is separated from other
simulation models. This separation makes the model extensible and it also allows
easy experimentation with alternative implementations. In a nutshell the power
model consists of the following components:

\begin{itemize}
  \item energy consumption models
  \item energy generation models
  \item temporary energy storage models
\end{itemize}

The following sections provide a brief overview of the power model.

\section{Energy Consumer Models}

These models describe the energy consumption of devices over time. For example,
a radio consumes energy when it transmits or receives signals, or a CPU consumes
energy when the network layer processes packets, or a display consumes energy
when it's turned on, etc. Energy consumers connect to an energy storage that
provides them with the required energy.

The physical layer provides a \nedtype{StateBasedEnergyConsumer} module that 
implements a simple radio energy consumption model. This model determines the
current power consumption of the radio using module parameters for each valid
combination of the radio mode, the transmitter state and the receiver state.

In order to support testing energy storage models a simple energy consumer model
is implemented in the \nedtype{AlternatingEnergyConsumer} module. This energy
consumer model alternates between two modes called consumption mode and sleep
mode. In consumption mode it consumes a randomly selected constant power for a
random time interval. In sleep mode it doesn't consume any energy for another
random time interval.
 
\section{Energy Generator Models}

These models describe the energy generation of devices over time. A solar panel,
for example, produces energy based on time, the panel's position on the globe,
its orientation towards the sun and the actual weather conditions. Energy
generators connect to an energy storage that absorbs the generated energy. 

In order to support testing energy storage models a simple energy generator
model is implemented in the \nedtype{AlternatingEnergyGenerator} module.
This energy generator model alternates between two modes called generation mode
and sleep mode. In generation mode it generates a randomly selected constant
power for a random time interval. In sleep mode it doesn't generate any energy
for another random time interval.
 
\section{Energy Storage Models}

These models describe devices that absorb energy produced by generators, and
provide energy for consumers. For example, an electrochemical battery in a
mobile phone provides energy for its display, its CPU, and its communication
devices. It might also absorb energy produced by a solar installed on its
display, or by a portable charger plugged into the wall socket.

The \nedtype{SimpleEnergyStorage} implements an energy storage model that is
similar to a battery. It computes its residual capacity by integrating the
difference between the total absorbed power and the total provided power over
time. It can initiate node shutdown when the residual capacity drops below a
configured threshold. It can also initiate node start when the residual capacity
raises above another configured threshold. Although this model is similar to a
real word battery it lacks some important properties such memory effect,
self-discharge, overcharging, temperature dependence, and so on.


\section{///////// NEW STUFF /////////}

\section{Overview}

% Why power consumption modeling is needed?
Modeling power consumption is essential, for example, for mobile personal
medical devices, large scale wireless environment monitoring devices, or
low-power wireless sensors. High fidelity simulation of power consumption
allows designing power sensitive routing protocols, MAC protocols with
power management features, which in turn results in more energy efficient
devices.

In order to help the modeling process, the INET power model is separated
from other simulation models. This separation makes the power model
extensible and it also allows easy experimentation with alternative
implementations. In a nutshell, the power model consists of energy
consumer, energy generator, energy storage and energy management models.

The power model elements fall into two categories abbreviated with \nedtype{Ep}
and \nedtype{Cc} as part of their names. Simpler models deal with energy and
power quantities, more realistic models deal with charge, current, and
voltage quantities.

\section{Energy Consumer Models}

% Why energy consumer models are needed?
In a network simulation, there are many components that in some way consume
energy. For example, a transceiver consumes energy when it transmits or
receives signals, a CPU consumes energy when the network protocol routes
packets, a display consumes energy when it’s turned on.

In INET, an energy consumer model is an OMNeT++ simple module implementing
the energy consumption of software processes or hardware devices over time.
Its main purpose is to provide the power or current consumption for the
current simulation time. Most often energy consumers are included as
submodules in the compound module of the hardware devices or software
components.

% What energy consumer models are provided?
INET provides only a few built-in energy consumer models:

\begin{itemize}
        \item \nedtype{AlternatingEpEnergyGenerator} is a trivial energy/power based statistical energy consumer model example.
        \item \nedtype{StateBasedEpEnergyConsumer} is a transceiver energy consumer model based on the radio mode and transmission/reception states.
\end{itemize}

In order to simulate power consumption in a wireless network, the energy
consumer model type must be configured for the transceivers. The following
example demonstrates how to configure the power consumption parameters for
a transceiver energy consumer model:

\inisnippet{EnergyConsumerConfigurationExample}{Energy consumer configuration example}

\section{Energy Generator Models}

% Why energy generator models are needed?
Its often necessary to also have components which actually generate energy
instead of consuming it, otherwise all energy storages become depleted
quickly and the simulation stops. In a wireless sensor network, a solar
panel, for example, produces energy based on time of day, the panel’s
location on the globe, its orientation towards the sunm and the actual
weather conditions.

In INET, an energy generator model is an OMNeT++ simple module implementing
the energy generation of a hardware device using a physical phenomena over
time. Its main purpose is to provide the power or current generation for
the current simulation time. Most often energy generation models are
included as submodules in network nodes.

INET provides only one trivial energy/power based statistical energy
generator model called \nedtype{AlternatingEpEnergyGenerator}. The following
example shows how to configure its power generation parameters:

\inisnippet{EnergyGeneratorConfigurationExample}{Energy generator configuration example}

\section{Energy Storage Models}

% Why energy storage models are needed?
Electronic devices which are not connected to the power grid contain some
component to store energy. For example, an electrochemical battery in a
mobile phone provides energy for its display, its CPU, and its
communication devices. It might also absorb energy produced by a solar
panel installed on its display, or by a portable charger plugged into the
wall socket every once in a while.

In INET, an energy storage model is an OMNeT++ simple module which models
the physical phenomena that is used to store energy produced by generators
and provide energy for consumers. Its main purpose is to compute the amount
of available energy or charge at the current simulation time. It maintains
a set of connected energy consumers and energy generators, their respective
total power consumption and total power generation.

INET contains a few built-in energy storage models:

\begin{itemize}
        \item \nedtype{IdealEpEnergyStorage} is an idealistic model with infinite energy capacity and infinite power flow.
        \item \nedtype{SimpleEpEnergyStorage} is a non-trivial model integrating the difference between the total consumed power and the total generated power over time.
        \item \nedtype{SimpleCcBattery} is a more realistic charge/current based battery model using a charge independent ideal voltage source and internal resistance.
\end{itemize}

The following example shows how to configure a simple energy storage model:

\inisnippet{EnergyStorageConfigurationExample}{Energy storage configuration example}

\section{Energy Management Models}

% Why energy management models are needed?
% The energy management models monitors an energy storage, estimates its state, and controls the consumers and generators to protect the energy storage from operating outside its safe operating area.

\nedtype{SimpleEpEnergyManagement}

\inisnippet{EnergyManagementConfigurationExample}{Energy management configuration example}



%%% Local Variables:
%%% mode: latex
%%% TeX-master: "usman"
%%% End:

