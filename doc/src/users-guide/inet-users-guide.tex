\documentclass{book}
\usepackage{a4wide}

%% possible fonts -- in order of preference
%%\usepackage{palatino}
\usepackage{bookman}
%%\usepackage{charter}
%%\usepackage{newcent}
%%\usepackage{times}
%%\usepackage{avant}
%%\usepackage{helvet}
%%\usepackage{sans}
%%\usepackage{chancery}

\usepackage[T1]{fontenc}
\usepackage{setspace}
\usepackage{ifpdf}
\usepackage{makeidx}
\usepackage{longtable}  %% page wrapping table environment
\usepackage{colortbl}   %% colors for tables
\usepackage{fancyvrb}   %% the "Verbatim" environment
\usepackage{fancyhdr}   %% custom headers and footers
\usepackage{multicol}
\usepackage{listings}   %% source code listings with syntax highlight (lstxxx commands)
\usepackage[tight]{shorttoc}   %% for generating a second table of contents, only containing chapter titles
\usepackage{bytefield}  %% for drawing protocol frames
\usepackage{paralist}   %% for compact lists
\usepackage[nottoc]{tocbibind}  %% makes Bibliography and Index show up in TOC
\settocbibname{References}

\setlength{\textwidth}{160mm}
%\setlength{\oddsidemargin}{12.5mm}
%\setlength{\evensidemargin}{12.5mm}
%\setlength{\topmargin}{0mm}
\setlength{\textheight}{220mm}
%\setlength{\parskip}{1ex}
%\setlength{\parindent}{5ex}

\renewcommand{\bottomfraction}{0.9}
\renewcommand{\topfraction}{0.9}
\renewcommand{\floatpagefraction}{0.9}

%% try to cure overfull hboxes
%% \tolerance=500

%% for navigation in dvi files, only needed by old teTeX versions
%%\usepackage{srcltx}

%% try this for spell checking: cat ess2002.tex | ispell -l -t -a | sort | uniq | more

%%
%% The following snippet changes the horizontal spacing between the number and
%% the title in the table of contents.
%%
%% http://tex.stackexchange.com/questions/33841/how-to-modify-the-space-between-the-numbers-and-text-of-sectioning-titles-in-the
%%
\makeatletter
 \renewcommand*\l@section{\@dottedtocline{1}{2em}{3em}}
 \renewcommand*\l@subsection{\@dottedtocline{2}{5em}{4em}}
\renewcommand*\l@chapter[2]{%
  \ifnum \c@tocdepth >\m@ne
    \addpenalty{-\@highpenalty}%
    \vskip 1.0em \@plus\p@
    \setlength\@tempdima{2em}%
    \begingroup
      \parindent \z@ \rightskip \@pnumwidth
      \parfillskip -\@pnumwidth
      \leavevmode \bfseries
      \advance\leftskip\@tempdima
      \hskip -\leftskip
      #1\nobreak\hfil \nobreak\hb@xt@\@pnumwidth{\hss #2}\par
      \penalty\@highpenalty
    \endgroup
  \fi}
\makeatother

%%
%% OMNeT++ logo, use as {\opp}
%%
\makeatletter
%%\DeclareRobustCommand{\omnetpp}{OM\-NeT\kern-.18em++\@}
\DeclareRobustCommand{\omnetpp}{OMNeT++\@}
\makeatother

\newcommand{\opp}{\omnetpp}

%%
%% PDF Header
%%
% note: \ifpdf now comes from the ifpdf package
%\newif\ifpdf
%\ifx\pdfoutput\undefined
%  \pdffalse
%\else
%  \pdfoutput=1
%  \pdftrue
%\fi
%% PDF-Info
\ifpdf
  \usepackage[pdftex]{graphicx}
  \usepackage[plainpages=false,linktocpage,bookmarksnumbered=true,pdftex]{hyperref}   %% automatic hyperlinking
  \pdfcompresslevel=9
  \pdfinfo{/Author (Andras Varga and others)
    /Title (INET Framework Users Guide)
    /Subject ()
    /Keywords (INET, INETMANET, OMNeT++, manual)}
\else
  \usepackage{graphicx}
  \usepackage[plainpages=false]{hyperref}   %% automatic hyperlinking
\fi

%%
%% Draft conditional to include unfinished parts
%%
\newif\ifdraft
%\draftfalse %% uncomment for final version
\drafttrue %% uncomment for draft version

%%
%% Generate Index
%%
\makeindex


%%
%% Link colors (hyperref package)
%%
\definecolor{MyDarkBlue}{rgb}{0.16,0.16,0.5}
%% XXX the next line apparently screws up all links except in TOC! they'll be colored nicely, but won't work.
%\hypersetup{
%    colorlinks=true,
%    linkcolor=MyDarkBlue,
%    anchorcolor=MyDarkBlue,
%    citecolor=MyDarkBlue,
%    filecolor=MyDarkBlue,
%    menucolor=MyDarkBlue,
%    runcolor=MyDarkBlue,
%    urlcolor=blue,
%}

%%
%% Heading and Footer
%%
\pagestyle{fancy}
\fancyhf{}
\renewcommand{\footrulewidth}{0.5pt}
\renewcommand{\chaptermark}[1]{\markboth{#1}{}}
\lhead{INET Framework Users Guide -- \leftmark}
\rfoot{\thepage}

%% this is used for chapter start pages
\fancypagestyle{plain}{
    \rfoot{\thepage}
}

%%
%% Use \begin{graybox}...\end{graybox} for notes
%%
\definecolor{MyGray}{rgb}{0.85,0.85,1.0}
\makeatletter\newenvironment{graybox}%
   {\begin{flushright}\begin{lrbox}{\@tempboxa}\begin{minipage}[r]{0.95\textwidth}}%
   {\end{minipage}\end{lrbox}\colorbox{MyGray}{\usebox{\@tempboxa}}\end{flushright}}%
\makeatother


\newenvironment{note}{\begin{graybox}\textbf{NOTE: }}{\end{graybox}}
\newenvironment{warning}{\begin{graybox}\textbf{WARNING: }}{\end{graybox}}
\newenvironment{caution}{\begin{graybox}\textbf{CAUTION: }}{\end{graybox}}
\newenvironment{rationale}{\begin{graybox}\textbf{Rationale: }}{\end{graybox}}
\newenvironment{important}{\begin{graybox}\textbf{IMPORTANT: }}{\end{graybox}}

%%
%% Set up listings package
%%
\lstloadlanguages{C++,make,perl,tcl,XML,R,Matlab}

%% See listings.pdf,pp20
\lstdefinelanguage{NED} {
    morekeywords={allowunconnected,bool,channel,channelinterface,connections,const,
                  default,double,extends,false,for,gates,if,import,index,inout,input,
                  int,like,module,moduleinterface,network,output,package,parameters,
                  property,simple,sizeof,string,submodules,this,true,types,volatile,
                  xml,xmldoc},
    sensitive=true,
    morecomment=[l]{//},
    morestring=[b]",
}
\lstdefinelanguage{MSG} {
    morekeywords={abstract,bool,char,class,cplusplus,double,enum,extends,false,
                  fields,int,long,message,namespace,noncobject,packet,properties,
                  readonly,short,string,struct,true,unsigned},
    sensitive=true,
    morecomment=[l]{//},
    morestring=[b]",
}
\lstdefinelanguage{inifile} {
    morekeywords={},
    sensitive=true,
    morecomment=[l]{\#},
    morestring=[b]",
}
\lstdefinelanguage{pseudocode} {
    morekeywords={if,then,else,otherwise,whenever,while},
    sensitive=true,
    morecomment=[l]{//},
    morestring=[b]",
    mathescape=true,
}

%% thick ruler on the left; also, designate backtick as LaTeX escape character
%% (e.g. \opp needs to be written as `\opp` inside listing blocks)
\lstset{
    escapechar=`,
    basicstyle=\ttfamily,
    showstringspaces=false,
    frame=leftline,
    framesep=10pt,
    framerule=3pt,
    xleftmargin=15pt
}

\definecolor{NEDRulerColor}{rgb}{0.5,1.0,0.5}  % pale green
\definecolor{MSGRulerColor}{rgb}{0.5,1.0,0.5}  % pale green
\definecolor{CPPRulerColor}{rgb}{0.8,0.5,0.2}  % pale orange
\definecolor{IniRulerColor}{rgb}{0.9,0.9,0.3}  % pale yellow
\definecolor{FileListingRulerColor}{rgb}{0.85,0.85,0.85}  % grey
%\definecolor{CommandLineRulerColor}{rgb}{0.9,0.9,0.2}
\definecolor{PseudoCodeRulerColor}{rgb}{0.0,1.0,1.0}  % cyan
\definecolor{XMLRulerColor}{rgb}{0.8,0.8,1.0}  % pale blue

%% See listings.pdf,pp39
\lstnewenvironment{ned}
    {\lstset{language=NED,rulecolor=\color{NEDRulerColor}}}
    {}
\lstnewenvironment{msg}
    {\lstset{language=MSG,rulecolor=\color{MSGRulerColor}}}
    {}
\lstnewenvironment{cpp}
    {\lstset{language=C++,rulecolor=\color{CPPRulerColor}}}
    {}
\lstnewenvironment{inifile}
    {\lstset{language=inifile,rulecolor=\color{IniRulerColor}}}
    {}
\lstnewenvironment{filelisting}
    {\lstset{language={},rulecolor=\color{FileListingRulerColor}}}
    {}
\lstnewenvironment{commandline}
    {\lstset{language={},framesep=11pt,framerule=1pt,xleftmargin=16pt}}
    {}
\lstnewenvironment{pseudocode}
    {\lstset{language=pseudocode,rulecolor=\color{PseudoCodeRulerColor}}}
    {}
\lstnewenvironment{XML}
    {\lstset{language=XML,rulecolor=\color{XMLRulerColor}}}
    {}

% add caption={#2} to display caption
\newcommand{\xmlsnippet}[2]{%
    \lstinputlisting[language=XML,rulecolor=\color{XMLRulerColor},linerange=<!\-\-#1\-\->-<!\-\-End\-\->,includerangemarker=false,firstnumber=0]{Snippets.xml}}
\newcommand{\cppsnippet}[2]{%
    \lstinputlisting[language=C++,rulecolor=\color{CPPRulerColor},linerange=//!#1-//!End,includerangemarker=false,firstnumber=0]{Snippets.cc}}
\newcommand{\msgsnippet}[2]{%
    \lstinputlisting[language=msg,rulecolor=\color{MSGRulerColor},linerange=//!#1-//!End,includerangemarker=false,firstnumber=0]{Snippets.msg}}
\newcommand{\nedsnippet}[2]{%
    \lstinputlisting[language=ned,rulecolor=\color{NEDRulerColor},linerange=//!#1-//!End,includerangemarker=false,firstnumber=0]{Snippets.ned}}
\newcommand{\inisnippet}[2]{%
    \lstinputlisting[language=inifile,rulecolor=\color{IniRulerColor},linerange=\#!#1-\#!End,includerangemarker=false,firstnumber=0]{Snippets.ini}}

%%
%% some customization
%%
\setlength{\parindent}{0pt}
\setlength{\parskip}{1ex}

%%
%% Shortcuts
%%
\newcommand{\appendixchapter}{\chapter} %% html converter needs to know which chapters are appendices

\newcommand{\tbf}{\textbf} %% bold faced text
\newcommand{\ttt}{\texttt} %% type writer font text

\newcommand{\tab}{\hspace*{5mm}} %% tabulator settings

\newcommand{\new}{$^{New!}$}
\newcommand{\changed}{$^{Changed!}$}

%% Colordefinition for table header rows (requires package colortbl)
\newcommand{\tabheadcol}{\rowcolor[gray]{0.8}}

%%
%% Module parameters list
%%
\newenvironment{params}{\begin{itemize}}{\end{itemize}}
\newcommand{\param}[2]{\item \fpar{#1}: #2}

%%
%% Function/Class/Macro/Variable/Program/Parameter/Define names
%%
%% Write the names in type writer font and do an index entry
%% Allows word wrap by automatic hyphenation
%%
%% Usage: \ffunc{take()}
%%    or: \ffunc[take()]{take(obj)}
%% the second form uses the bracketed word for the index entry
%%

\newcommand{\protocol}[1]{%
    {#1}}

%% NED type names
\newcommand{\nedtype}[2][\DefaultOpt]{\def\DefaultOpt{#2}%
  \index{#1}%
  \texttt{\hyphenchar\font=`\-\relax#2}}

%% MSG type names
\newcommand{\msgtype}[2][\DefaultOpt]{\def\DefaultOpt{#2}%
  \index{#1}%
  \texttt{\hyphenchar\font=`\-\relax#2}}

%% Function names
\newcommand{\ffunc}[2][\DefaultOpt]{\def\DefaultOpt{#2}%
  \index{#1}%
  \texttt{\hyphenchar\font=`\-\relax#2}}

%% Class names
\newcommand{\cppclass}[2][\DefaultOpt]{\def\DefaultOpt{#2}%
  \index{#1}%
  \texttt{\hyphenchar\font=`\-\relax#2}}

%% Macro names
\newcommand{\fmac}[2][\DefaultOpt]{\def\DefaultOpt{#2}%
  \index{#1}%
  \texttt{\hyphenchar\font=`\-\relax#2}}

%% Variable names
\newcommand{\fvar}[2][\DefaultOpt]{\def\DefaultOpt{#2}%
  \index{#1}%
  \texttt{\hyphenchar\font=`\-\relax#2}}

%% Program names
\newcommand{\fprog}[2][\DefaultOpt]{\def\DefaultOpt{#2}%
  \index{#1}%
  \texttt{\hyphenchar\font=`\-\relax#2}}

%% Parameter names
\newcommand{\fpar}[2][\DefaultOpt]{\def\DefaultOpt{#2}%
  \index{#1}%
  \texttt{\hyphenchar\font=`\-\relax#2}}

%% Defines
\newcommand{\fdef}[2][\DefaultOpt]{\def\DefaultOpt{#2}%
  \index{#1}%
  \texttt{\hyphenchar\font=`\-\relax#2}}

%% NED/MSG properties
\newcommand{\fprop}[2][\DefaultOpt]{\def\DefaultOpt{#2}%
  \index{#1}%
  \texttt{\hyphenchar\font=`\-\relax#2}}

%% Keywords (NED, MSG)
\newcommand{\fkeyword}[2][\DefaultOpt]{\def\DefaultOpt{#2}%
  \index{#1}%
  \textbf{\texttt{\hyphenchar\font=`\-\relax#2}}}

%% Configuration options
\newcommand{\fconfig}[2][\DefaultOpt]{\def\DefaultOpt{#2}%
  \index{#1}%
  \textbf{\texttt{\hyphenchar\font=`\-\relax#2}}}

%% File names
\newcommand{\ffilename}[2][\DefaultOpt]{\def\DefaultOpt{#2}%
  \index{#1}%
  \texttt{\hyphenchar\font=`\-\relax#2}}

%% Signals
\newcommand{\fsignal}[2][\DefaultOpt]{\def\DefaultOpt{#2}%
  \index{#1}%
  \texttt{\hyphenchar\font=`\-\relax#2}}

\newcommand{\fgate}[1]{\texttt{\hyphenchar\font=`\-\relax#1}}

%% do not number subsubsections
%\setcounter{secnumdepth}{4}

% limit the depth of TOC
\setcounter{tocdepth}{2}

%%
%% Start of document
%%
\begin{document}

%% set the image type preference
\DeclareGraphicsExtensions{.pdf,.png}

\pagestyle{empty}
\pagenumbering{roman}
\include{title}
\cleardoublepage

%%\setcounter{page}{1}
%\newpage
%%\pagenumbering{roman}

%% \shorttableofcontents{Chapters}{0}
%% \cleardoublepage

\tableofcontents
\cleardoublepage

\pagestyle{fancy}
\pagenumbering{arabic}

\include{ch-introduction}
\cleardoublepage

\include{ch-usage}
\cleardoublepage

\include{ch-networks}
\cleardoublepage

\include{ch-network-nodes}
\cleardoublepage

\chapter{Network Interafces}
\label{cha:network-interfaces}

\section{Overview}

TODO

\section{The Interface Table}

The \nedtype{InterfaceTable} module holds one of the key data structures in
the INET Framework: information about the network interfaces in the host.
The interface table module does not send or receive messages; other modules
access it using standard C++ member function calls.

\ifdraft TODO
\subsection{Accessing the Interface Table}

If a module wants to work with the interface table, first it needs to obtain a
pointer to it. This can be done with the help of the
\cppclass{InterfaceTableAccess} utility class:

\begin{cpp}
IInterfaceTable *ift = InterfaceTableAccess().get();
\end{cpp}

\cppclass{InterfaceTableAccess} requires the interface table module to be a
direct child of the host and be called \ttt{"interfaceTable"} in order to
be able to find it. The \ffunc{get()} method never returns \ttt{NULL}: if
it cannot find the interface table module or cannot cast it to the
appropriate C++ type (\cppclass{IInterfaceTable}), it throws an exception
and stop the simulation with an error message.

For completeness, \cppclass{InterfaceTableAccess} also has a
\ffunc{getIfExists()} method which can be used if the code does not require
the presence of the interface table. This method returns \ttt{NULL} if the
interface table cannot be found.

Note that the returned C++ type is \cppclass{IInterfaceTable}; the initial
"\ttt{I}" stands for "interface". \cppclass{IInterfaceTable} is an abstract
class interface that \cppclass{InterfaceTable} implements. Using the abstract
class interface allows one to transparently replace the interface table with
another implementation, without the need for any change or even
recompilation of the INET Framework.
\fi

\subsection{Interface Entries}

Interfaces in the interface table are represented with the
\cppclass{InterfaceEntry} class. \cppclass{IInterfaceTable} provides member
functions for adding, removing, enumerating and looking up interfaces.

Interfaces have unique names and interface IDs; either can be used to look up
an interface (IDs are naturally more efficient). Interface IDs are invariant to
the addition and removal of other interfaces.

Data stored by an interface entry include:

\begin{itemize}
  \item \textit{name} and \textit{interface ID} (as described above)
  \item \textit{MTU}: Maximum Transmission Unit, e.g. 1500 on Ethernet
  \item several flags:
    \begin{itemize}
      \item \textit{down}: current state (up or down)
      \item \textit{broadcast}: whether the interface supports broadcast
      \item \textit{multicast} whether the interface supports multicast
      \item \textit{pointToPoint}: whether the interface is point-to-point link
      \item \textit{loopback}: whether the interface is a loopback interface
    \end{itemize}
  \item \textit{datarate} in bit/s
  \item \textit{link-layer address} (for now, only IEEE 802 MAC addresses are supported)
  \item \textit{network-layer gate index}: which gate of the network layer within the host the NIC is connected to
  \item \textit{host gate IDs}: the IDs of the input and output gate of the host the NIC is connected to
\end{itemize}

\tbf{Extensibility}. You have probably noticed that the above list does not
contain data such as the IPv4 or IPv6 address of the interface. Such
information is not part of \cppclass{InterfaceEntry} because we do not want
\nedtype{InterfaceTable} to depend on either the IPv4 or the IPv6 protocol
implementation; we want both to be optional, and we want
\nedtype{InterfaceTable} to be able to support possibly other network
protocols as well.

Thus, extra data items are added to \cppclass{InterfaceEntry} via
extension. Two kinds of extensions are envisioned: extension by the link
layer (i.e. the NIC), and extension by the network layer protocol:

\begin{itemize}

\item \tbf{NICs} can extend interface entries via C++ class inheritance, that is, by
simply subclassing \cppclass{InterfaceEntry} and adding extra data and
functions. This is possible because NICs create and register entries in
\nedtype{InterfaceTable}, so in their code one can just write
\ttt{new MyExtendedInterfaceEntry()} instead of \ttt{new InterfaceEntry()}.

\item \textbf{Network layer protocols} cannot add data via subclassing, so
composition has to be used. \cppclass{InterfaceEntry} contains pointers to
network-layer specific data structures. For example, there are pointers to
IPv4 specific data, and IPv6 specific data. These objects can be accessed with
the following \cppclass{InterfaceEntry} member functions: \ffunc{ipv4Data()},
\ffunc{ipv6Data()}, and \ffunc{getGenericNetworkProtocolData()}.
They return pointers of the types \cppclass{Ipv4InterfaceData},
\cppclass{Ipv6InterfaceData}, and \cppclass{GenericNetworkProtocolInterfaceData},
respectively. For illustration, \cppclass{Ipv4InterfaceData} is installed
onto the interface entries by the \nedtype{Ipv4RoutingTable} module, and it
contains data such as the IP address of the interface, the netmask, link
metric for routing, and IP multicast addresses associated with the
interface. A protocol data pointer will be \ttt{NULL} if the corresponding
network protocol is not used in the simulation; for example, in IPv4
simulations only \ffunc{ipv4Data()} will return a non-\ttt{NULL} value.


\end{itemize}


\subsection{Interface Registration}

Interfaces are registered dynamically in the initialization phase by modules
that represent network interface cards (NICs). The INET Framework makes use
of the multi-stage initialization feature of OMNeT++, and interface registration takes
place in the first stage (i.e. stage \ttt{INITSTAGE\_LINK\_LAYER}).

Example code that performs interface registration:

\begin{cpp}
void PPP::initialize(int stage)
{
    if (stage == INITSTAGE_LINK_LAYER) {
        ...
        interfaceEntry = registerInterface(datarate);
    ...
}

InterfaceEntry *PPP::registerInterface(double datarate)
{
    InterfaceEntry *e = new InterfaceEntry(this);

    // interface name: NIC module's name without special characters ([])
    e->setName(OPP_Global::stripnonalnum(getParentModule()->getFullName()).c_str());

    // data rate
    e->setDatarate(datarate);

    // generate a link-layer address to be used as interface token for IPv6
    InterfaceToken token(0, simulation.getUniqueNumber(), 64);
    e->setInterfaceToken(token);

    // set MTU from module parameter of similar name
    e->setMtu(par("mtu"));

    // capabilities
    e->setMulticast(true);
    e->setPointToPoint(true);

    // add
    IInterfaceTable *ift = findModuleFromPar<IInterfaceTable>(par("interfaceTableModule"), this);
    ift->addInterface(e);

    return e;
}
\end{cpp}

\ifdraft TODO
\subsection{Interface Change Notifications}

\nedtype{InterfaceTable} has a change notification mechanism built in, with
the granularity of interface entries.

Clients that wish to be notified when something changes in
\nedtype{InterfaceTable} can subscribe to the following notification
categories in the host's \nedtype{NotificationBoard}:

\begin{itemize}
  \item \tbf{\ttt{NF\_INTERFACE\_CREATED}}: an interface entry has been
    created and added to the interface table
  \item \tbf{\ttt{NF\_INTERFACE\_DELETED}}: an interface entry is going
    to be removed from the interface table. This is a pre-delete
    notification so that clients have access to interface data that are
    possibly needed to react to the change
  \item \tbf{\ttt{NF\_INTERFACE\_CONFIG\_CHANGED}}: a configuration setting
    in an interface entry has changed (e.g. MTU or IP address)
  \item \tbf{\ttt{NF\_INTERFACE\_STATE\_CHANGED}}: a state variable in an
    interface entry has changed (e.g. the up/down flag)
\end{itemize}

In all those notifications, the data field is a pointer to the
corresponding \cppclass{InterfaceEntry} object. This is even true for
\ttt{NF\_INTERFACE\_DELETED} (which is actually a pre-delete notification).
\fi


%%% Local Variables:
%%% mode: latex
%%% TeX-master: "usman"
%%% End:


\cleardoublepage

\include{ch-ppp}
\cleardoublepage

\include{ch-ethernet}
\cleardoublepage

\chapter{The Physical Layer}
\label{cha:physicallayer}

\section{Overview}

Today's electric devices use more and more wireless communication methods such
as Wifi, Bluetooth, NFC, UMTS, and LTE. Despite the diversity of these devices
there are many similarities in the modeling of their physical layer components.
The models often have similar signal representations and signal processing steps,
and they also share the physical medium model where communication takes place.

In general, the physical layer simulation is a very time consuming task. The
simulation of signal propagation, signal fading, signal interference, and signal
decoding in detail may often result in unacceptable performance. Finding the
right abstractions, the right level of detail, and the right trade-offs between
accuracy and performance is difficult and very important.

To summarize, the physical layer is designed with the following goals in mind:
\begin{itemize}
  \item customizability
  \item extensibility
  \item scalable level of detail
  \item ability to exploit parallel hardware
\end{itemize}

The following sections provide a brief overview of the physical layer model. For
more details on the available modules, their parameterization and the actual
implementations please refer to the documentation in the corresponding NED and
C++ source files.

\subsection{Customizability}

Real world communication devices often provide a wide variety of configuration
options to allow adapting to the physical conditions where they are required
to operate. For example, a Wifi router administration interface often provides
parameters to configure the transmission power, bitrate, preamble type, carrier
frequency, RTS threshold, beacon interval, etc. Mostly these parameters have
default values assigned, so the user doesn't have to set them separately, but
may override them as needed.

Similarly to real world devices the physical layer models also provide a wide
variety of parameters to control their behavior. The most common NED parameters
are various physical quantities with physical units such as transmission power
\ttt{[W]}, reception sensitivity \ttt{[W]}, carrier frequency \ttt{[Hz]},
communication range \ttt{[m]}, propagation speed \ttt{[m/s]}, SNIR reception
threshold \ttt{[dB]}, bitrate \ttt{[b/s]}. Occasionally models support new
parameters or new combinations, which don't exist in real world hardware, to
allow further experimentation.

Another important and commonly used parameter kind selects among alternative
implementations of a particular interface by providing its name. Different
implementations are often separate modules, which come with their own set of
parameters to avoid the confusion of mixing their unrelated parameters. Some
modules may be split into more submodules. This further deepens the module
hierarchy, but allows better extensibility.

\subsection{Extensibility}

Similarly to designing other simulation models, modeling the physical layer is
not at all an unambiguous task. For example, the research literature contains a
number of different path loss models for signal propagation, there are different
bit error models for a particular protocol standard, representing the signal in
the analog domain can also be done in several different ways, and so on.

In order to support this diversity the physical layer is designed to be
extensible with alternative implementations at various parts of the model. This
is realized by separately defining C++ and NED interfaces between modules, and
also by providing parameters in their parent modules to easily select among the
available implementations.

New models can be added by implementing the required interfaces from scratch, or
by deriving from already existing implementations and overriding functionality.
This architecture allows the user to create new models with less effort, and to
focus on the real differences, while the rest of the physical layer remains the
same.

\subsection{Scalable Level of Detail}

There are many possible ways to model various aspects of the physical layer.
The most important difference lies in the trade-off between performance versus
accuracy. In order to support the different trade-offs the physical layer is
designed to be scalable with respect to the simulated level of detail. In other
words, it's scalable from high-performance less accurate simulations to high
fidelity slower simulations.

The physical layer model is scalable along the following axes:

\begin{itemize}
  \item simulation model
  \item software architecture
  \item data representation
  \item number of messages
\end{itemize}

The simulation model might vary from simple statistical models to accurate
emulation. The simplest models ignore the actual bits of the transmission. For
example, the extremely simple unit disc radio even ignores the signal power. The
most accurate models use precise signal representation for all four domains:
bit domain, symbol domain, sample domain, and analog domain representations.
They also emulate most functions of real hardware in detail: forward error
correction, interleaving, scrambling, modulation, spreading, pulse shaping, and
so on.

The software architecture might vary from flat to layered. A flat architecture
is efficient but not modular. Functionality can only be affected through simple
parameters and not by providing alternative implementations. Whereas a layered
architecture is more flexible at the cost of more complex data structures, more
data conversions, more resource management, and thus slower processing. On the
other hand, it provides more customization opportunities to replace parts with
alternative implementations and to do research easier in the area.

The data representation might vary from scalar to multidimensional values. In
the analog domain of the physical layer data quite often changes over time,
frequency, space, or any combination thereof. The most obvious example is the
analog signal power, but there are others such as signal phase or the signal to
noise ratio.

The number of messages per transmission added to the future event queue might
vary from one to the number of radios. One message might be sufficient, for
example, if the transmission is intended to a single destination, and other
receivers are either not affected, or the effect is negligible. On the other
hand, it might be necessary to process all transmissions by all receivers in
order to have the desired effect on the higher layers. For example, if a MAC
model is configured to promiscuous mode, it needs to receive all transmissions.

\subsection{Exploiting Parallel Hardware}

The physical processes simulated by the physical layer are inherently parallel.
The computation of the transmission arrival space-time coordinates, the analog
signal representation of transmissions and receptions, the interfering
receptions and noises, the signal to noise ratio, the decoded bits, the bit
errors, and the physical layer indications all provide a good parallelization
opportunity, because they dominate the physical layer performance and are
independent for each receiver. Therefore the physical layer is designed to be
able to utilize parallel hardware, multi-core CPUs, vector instructions and the
highly parallel GPU.

The idea is to have a central component in the software architecture where
parallel computation can happen. This central component is the medium model
that knows about all radios, transmissions, interferences, and receptions
anyway. It uses optimistic parallel computation in multiple background threads
while the main simulation thread continues normal execution. When a new
transmission enters the channel the already computed and affected results are
invalidated or updated, and the affected ongoing optimistic parallel
computations are canceled.

\section{The Radio Model}

The radio model describes the physical device that is capable of transmitting
and receiving signals on the medium. It contains an antenna model, a transmitter
model, a receiver model, and an energy consumer model. The antenna model is
shared between the transmitter model and the receiver model. The separation of
the transmitter model and the receiver model allows asymmetric configurations.
The energy consumer model is optional and it's only used when the simulation of
energy consumption is necessary.

The radio model has an operational mode that is called the radio mode. The radio
mode is externally controlled usually by the MAC model. In transceiver mode, the
radio can simultaneously transmit and receive a signal. Changing the radio mode
may optionally take a non-zero amount of time. The supported radio modes are the
following:

\begin{itemize}
  \item \ttt{off}: communication isn't possible, energy consumption is zero
  \item \ttt{sleep}: communication isn't possible, energy consumption is minimal
  \item \ttt{receiver}: only reception is possible, energy consumption is low
  \item \ttt{transmitter}: only transmission is possible, energy consumption is
high
  \item \ttt{transceiver}: reception and transmission is simultaneously
possible, energy consumption is high
  \item \ttt{switching}: communication isn't possible, energy consumption is
minimal
\end{itemize}

In addition to the radio mode, the transmitter and the receiver models have
separate states which describe what they are doing. Changes to these states are
automatically published by the radio. The signaled transmitter states are the
following:

\begin{itemize}
  \item \ttt{undefined}: isn't operating
  \item \ttt{idle}: there's no transmission in progress
  \item \ttt{transmitting}: transmission is in progress
\end{itemize}

The signaled receiver states are the following:

\begin{itemize}
  \item \ttt{undefined}: isn't operating
  \item \ttt{idle}: there's no reception in progress
  \item \ttt{busy}: received signal is not interpretable
  \item \ttt{synchronizing}: synchronization is in progress
  \item \ttt{receiving}: reception is in progress
\end{itemize}

When a radio wants to transmit a signal on the medium it sends direct messages
to all affected radios with the help of the central medium module. The messages
contain a shared data structure which describes the transmission the way it
entered the medium. The messages arrive at the moment when start of the
transmission arrive at the receiver. The receiver radios also handle the
incoming messages with the help of the central medium module. This kind of
centralization allows the medium to do shared computations in a more efficient
way and it also makes parallel computation possible.

As stated above the radio module utilizes multiple submodules to further split
its task. This design decision makes it more extensible and customizable. The
following sections describe the parts of the radio model.

\subsection{Antenna Models}

The antenna model describes the effects of the physical device which converts
electric signals into radio waves, and vice versa. This model captures the
antenna characteristics that heavily affect the quality of the communication
channel. For example, various antenna shapes, antenna size and geometry, antenna
arrays, and antenna orientation causes different directional or frequency
selectivity.

The antenna model provides a position and an orientation using a mobility model
that defaults to the mobility of the node. The main purpose of this model is to
compute the antenna gain based on the specific antenna characteristics and the
direction of the signal. The signal direction is computed by the medium from the
position and the orientation of the transmitter and the receiver. The following
list provides some examples:

\begin{itemize}
  \item \nedtype{IsotropicAntenna}: antenna gain is exactly 1 in any direction
  \item \nedtype{ConstantGainAntenna}: antenna gain is a constant determined by
a parameter
  \item \nedtype{DipoleAntenna}: antenna gain depends on the direction according
to the dipole antenna characteristics
  \item \nedtype{InterpolatingAntenna}: antenna gain is computed by linear
interpolation according to a table indexed by the direction angles
\end{itemize}

The antenna models are in the \ttt{src/physicallayer/antenna/} directory.

\subsection{Transmitter Models}

The transmitter model describes the physical process which converts packets into
electric signals. In other words, this model converts a MAC packet into a signal
that is transmitted on the medium. The conversion process and the representation
of the signal depends on the level of detail and the physical characteristics
of the implemented protocol.

In the flat model the transmitter model skips the symbol domain and the sample
domain representations, and it directly creates the analog domain representation.
The bit domain representation is reduced to the bit length of the packet and the
actual bits are ignored.

In the layered model the conversion process involves various processing steps
such as packet serialization, forward error correction encoding, scrambling,
interleaving, and modulation. This transmitter model requires much more
computation, but it produces accurate bit domain, symbol domain, and sample
domain representations.

The various protocol specific transmitter models are in the corresponding
directories.

\subsection{Receiver Models}

The receiver model describes the physical process which converts electric
signals into packets. In other words, this model converts a reception, along
with an interference computed by the medium model, into a MAC packet and a
reception indication. It also determines the following for each transmission: 

\begin{itemize}
  \item \ttt{is the reception possible or not}: based on the signal
characteristics such as reception power, carrier frequency, bandwidth, preamble
mode, modulation scheme
  \item \ttt{if the reception is possible, is reception attempted or not}: based
on the ongoing reception and the support of signal capturing
  \item \ttt{if the reception is attempted, is reception successful or not}:
based on the error model and the simulated part of the signal decoding
\end{itemize}

In the flat model the receiver model skips the sample domain, the symbol domain,
and the bit domain representations, and it directly creates the packet domain
representation by copying the packet from the transmission. It uses the error
model to decide if the reception is successful or not.

In the layered model the conversion process involves various processing steps
such as demodulation, descrambling, deinterleaving, forward error correction
decoding, and deserialization. This reception model requires much more
computation, but it produces accurate sample domain, symbol domain, and bit
domain representations.

The various protocol specific receiver models are in the corresponding 
directories.

\subsection{Transmission Error Modeling}

Determining the reception errors is a crucial part of the reception process.
There are often several different statistical error models in the literature
even for a particular physical layer. In order to support this diversity the
error model is a separate replaceable component of the receiver. 

The error model describes how the signal to noise ratio affects the amount of
errors at the receiver. The main purpose of this model is to determine whether
if the received packet has errors or not. It also computes various physical
layer indications for higher layers such as packet error rate, bit error rate,
and symbol error rate. For the layered reception model it needs to compute the
erroneous bits, symbols, or samples depending on the lowest simulated physical
domain where the real decoding starts. The error model is optional, if omitted
all receptions are considered successful.

The error models are in the \ttt{src/physicallayer/errormodel/} directory and
also in the corresponding protocol specific directories.

\subsection{Power Consumption Models}

A substantial part of the energy consumption of communication devices comes from
transmitting and receiving signals. The energy consumer model describes how the
radio consumes energy depending on its activity. This model is optional, if
omitted energy consumption is ignored. The following list provides some examples:

\begin{itemize}
  \item \nedtype{StateBasedEnergyConsumer}: the constant power consumption is
determined by valid combinations of the radio mode, the transmitter state and
the receiver state
\end{itemize}

The energy consumer models are in the \ttt{src/physicallayer/energyconsumer/} directory.

\ifdraft
TODO: layered
\subsection{Layered Radio Model}

This module further splits the transmitter and receiver models to allow bit
precise communication modeling.

TODO: layered

The following sections describe the parts of the layered radio model.

\subsubsection{Encoding and Decoding}

This module describes how the packet domain signal representation is converted
into the bit domain, and vice versa.

TODO: layered

\subsubsection{Modulation and Demodulation}

This module describes how the bit domain signal representation is converted into
the symbol domain, and vice versa.

TODO: layered

\subsubsection{Pulse Shaping and Pulse Filtering}

This module describes how the symbol domain signal representation is converted
into the sample domain, and vice versa.

TODO: layered


\subsubsection{Digital Analog and Analog Digital Conversion}

This module describes how the sample domain signal representation is converted
into the analog domain, and vice versa.

TODO: layered
\fi

\section{The Medium Model}

The medium model describes the shared physical medium where communication takes
place. It keeps track of radios, noise sources, ongoing transmissions,
background noise, and other ongoing noises. The medium computes when, where and
how transmissions and noises arrive at receivers. It also efficiently provides
the set of interfering transmissions and noises for the receivers. It doesn't
send or handle messages on its own, it rather acts as a mediator between radios.

The medium model has a separate chapter devoted to it, see \ref{cha:transmission-medium}. 

\section{Signal Representation}

The data structures that represent the transmitted and the received signals
might contain many different data depending on the simulated level of detail. In
addition, the reception data structure might contain various physical layer
indications, which are computed during the reception process. The following list
provides some examples:

\begin{itemize}
  \item \ttt{packet domain}: actual packet, packet error rate, packet error bit,
etc.
  \item \ttt{bit domain}: various bit lengths, bitrates, actual bits, forward
error correction code, interleaving scheme, scrambling scheme, bit error rate,
number of bit errors, actual erroneous bits, etc.
  \item \ttt{symbol domain}: number of symbols, symbol rate, actual symbols,
modulation scheme, symbol error rate, number of symbol errors, actual erroneous
symbols, etc.
  \item \ttt{sample domain}: number of samples, sampling rate, actual samples,
etc.
  \item \ttt{analog domain}: space-time coordinates, antenna orientations,
communication range, interference range, detection range, carrier frequency,
subcarrier frequencies, bandwidths, scalar or dimensional power, receive signal
strength indication, signal to noise and interference ratio, etc.
\end{itemize}

In simple case the packet domain specifies the MAC packet only, and the bit
domain specifies the bit length and the bitrate. The symbol domain specifies the
used modulation, and the sample domain is simply ignored. The most important
part is the analog domain representation, because it's indispensable to be able
to compute some kind of signal to noise and interference ratio. The following
figure shows four different kinds of analog domain representations, but other
representations are also possible.

\begin{figure}[h!]
\centering
\includegraphics[width=\textwidth]{figures/phyanalog}
\caption{Various analog signal representations}
\end{figure}

The first representation is called range-based, and it's used by the unit disc
radio. The advantage of this data structure is that it's compact, predictable,
and provides high performance. The disadvantage is that it's very inaccurate in
terms of modeling reality. Nevertheless, this representation might be sufficient
for developing a new routing protocol if accurate simulation of packet loss is
not important.

The second data structure represents a narrowband signal with a scalar signal
power, a carrier frequency, and a bandwidth. The advantage of this
representation is that it allows to compute a real signal to noise ratio, which
in turn can be used by the error model to compute bit and packet error rates.
This representation is most of the time sufficient for the simulation of IEEE
802.11 networks.

The third data structure describes a signal power that changes over time. In
this case the signal power is represented with a one-dimensional time dependent
value that precisely follows the transmitted pulses. This representation is used
by the IEEE 802.15.4a UWB radio.

The last representation uses a multi-dimensional value to describe the signal
power that changes over both time and frequency. The IEEE 802.11b model might
use this representation to simulate crosstalk, where one channel interferes with
another. In order to make it work the frequency spectrum of the signal has to
follow the real spectrum more precisely at both ends of the band.

The flat signal representation uses a single object to simulatenously describe
all domains of the transmission or the reception. In contrast, the layered
signal representation uses one object to describe every domain seperately. The
advantage of the latter is that it's extensible with alternative implementations
for each domain. The disadvantage is that it needs more allocation and resource
management.

\section{Signal Processing}

The following figure shows the process of how a MAC packet gets from the
transmitter radio through the medium to the receiver radio. The figure focues on
how data flows between the processing components of the physical layer. The blue
boxes represent the data structures, and the red boxes represent the processing
components.

\begin{figure}[h!]
\centering
\includegraphics[width=\textwidth]{figures/phydataflow}
\caption{Signal processing data flow}
\end{figure}

The transmission process starts in the transmitter radio when it receives a MAC
packet from the higher layer. The radio must be in transmitter or transceiver
mode before receiving a MAC packet, otherwise it throws an exception. At first
the transmitter model creates a data structure that describes the transmitted
signal based on the received MAC packet and the attached transmission request.
The resulting data structure is immutable, it's not going to be changed in any
later processing step.

Thereafter the propagation model computes the arrival space-time coordinates for
all receivers. In the next step the medium model determines the set of affected
receivers. Which radio constitutes affected depends on a number of factors such
as the maximum communication range of the transmitter, the radio mode of the
receiver, the listening mode of the receiver, or potentially the MAC address of
the receiver. Using the result the medium model sends a separate message with
the shared transmission data structure to all affected receivers. There's no
need to send a message to all radios on the channel, because the computation
of interfering signals is independent of this step.

Thereafter the attenuation model computes the reception for the receiver using
the original transmission and the arrival data structure. It applies the path
loss model, the obstacle loss model and the multipath model to the transmission.
The resulting data structure is also immutable, it's not going to be changed in
any later processing step.

Thereafter the medium model computes the interference for the reception by
collecting all interfering receptions and noises. Another signal is considered
interfering if it owerlaps both in time and frequency domains with respect to 
the minimum interference parameters. The background noise model also computes a 
noise signal that is added to the interference.

The reception process starts in the receiver radio when it receives a message
from the transmitter radio. The radio must be in receiver or transceiver mode
before the message arrives, otherwise it ignores the message. At first the
receiver model determines is whether the reception is actually attempted or not.
This decision depends on the reception power, whether there's another ongoing
reception process, and capturing is enabled or not.

Thereafter the receiver model computes the signal to noise and interference
ratio from the reception and the interference. Using the result, the bitrate,
and the modulation scheme the error model computes the necessary error rates.
Alternatively the error model might compute the erroneous bits, or symbols by
altering the corresponding data of the original transmission. 

Thereafter the receiver determines the received MAC packet by either simply
reusing the original, or actually decoding from the lowest represented domain
in the reception. Finally, it attaches the physical layer indication to the MAC
packet, and sends it up to the higher layer.

The following sections describe the data structures that are created during
signal processing.

\subsubsection{Transmission Request}

This data structure contains parameters that control how the transmitter
produces the transmission. For example, it might override the default
transmission power, ot the default bitrate of the transmitter. It is attached as
a control info object to the MAC packet sent down from the MAC module to the
radio.

\subsubsection{Transmission}

This data structure describes the transmission of a signal. It specifies the
start/end time, start/end antenna position, start/end antenna orientation of the
transmitter. In other words, it describes when, where and how the signal
started/ended to interact with the medium. The transmitter model creates one
transmission instance per MAC packet.

\subsubsection{Arrival}

This data structure decscirbes the space and time coordinates of a transmission
arriving at a particular receiver. It specifies the start/end time, start/end
antenna position, start/end antenna orientation of the receiver. The propagation
model creates one arrival instance per transmission per receiver.

\subsubsection{Listening}

This data structure describes the way the receiver radio is listening on the
medium. The physical layer ignores certain transmissions either during computing
the interference or even the complete reception of such transmissions. For
example, a narrowband listening specifies a carrier frequency and a bandwidth. 

\subsubsection{Reception}

This data structure describes the reception of a signal by a particular receiver.
It specifies at least the start/end time, start/end antenna position, start/end
antenna orientation of the receiver. The attenuation model creates one reception
instance per transmission per receiver.

\subsubsection{Noise}

This data structure describes a meaningless signal or a meaningless composition
of multiple signals. All models contain at least the start/end time, and
start/end position.

\subsubsection{Interference}

This data structure describes the interfering signals and noises that affect a
particular reception. It also specifies the total noise that is the composition
of all interference.

\subsubsection{SNIR}

This data structure describes the signal to noise and interference ratio of a
particular reception. It also specifies the minimum signal to noise and
interference ratio.

\subsubsection{Reception Decision}

This data structure describes whether if the reception of a signal is possible
or not, is attempted or not, and is successful or not.

\subsubsection{Reception Indication}

This data structure describes the physical layer indications such as RSSI, SNIR,
PER, BER, SER. These physical properties are optional and may be omitted if the
receiver is configured to do so or if it doesn't support providing the data. The
reception indication is attached as a control info object to the MAC packet sent
up from the radio to the MAC module. 

\section{Visualization}

In order to help understanding the communication in the network the physical
layer supports visualizing its state. The following list shows what can be
displayed:

\begin{itemize}
  \item ongoing transmissions
  \item recent successful receptions
  \item recent obstacle intersections and surface normal vectors
\end{itemize}

The ongoing transmissions can be displayed with 3 dimensional spheres or with 2
dimensional rings laying in the XY plane. As the signal propagates through space
the figure grows with it to show where the beginning of the signal is. The inner
circle of the ring figure shows as the end of the signal propagates through
space. 

The recent successful receptions are displayed as straight lines between the
original positions of the transmission and the reception. The recent obstacle
intersections are also displayed as straight lines from the start of the
intersection to the end of it.

\ifdraft
\section{TODO other stuff}

TODO: scalar vs dimensional

TODO: flat vs layered

TODO: Generic, IEEE 802.11, IEEE 802.15.4

TODO: acoustic underwater example

TODO: wireless vs. wired medium

\section{Use Cases}
\fi


%%% Local Variables:
%%% mode: latex
%%% TeX-master: "usman"
%%% End:


\cleardoublepage

\chapter{Modeling the Wireless Channel (alias: The Transmission Medium)}
\label{cha:transmission-medium}

\section{Overview}

For wireless communication, an additional module is required to model the
shared physical medium where the communication takes place. This module
keeps track of transceivers, noise sources, ongoing transmissions,
background noise, and other ongoing noises.

The transmission medium is modeled as an OMNeT++ compound module with
several replaceable submodules. It contains submodules to model signal
propagation, path loss, obstacle loss, signal analog model, background
noise, and various caches for efficiency. With the help of its submodules,
the medium module computes when, where, and how signals arrive at
receivers, including the set of interfering signals and noises.

As a central component, the medium module influences performance to a large
extent, so it also provides a couple of parameters for optimization. For
example, the \nedtype{rangeFilter} parameter controls how the set of affected
receivers is determined based on their distance when the signal enters the
medium.

\section{Propagation Models}

When a transmitter starts to transmit a signal, the beginning of the signal
propagates through the transmission medium. When the transmitter ends the
transmission, the signal's end propagates similarly. The propagation model
describes how a signal moves through space over time. Its main purpose is
to compute the arrival space-time coordinates at receivers. There are two
built-in models in INET, implemented as simple modules:

\begin{itemize}
        \item \nedtype{ConstantTimePropagation} is a simplistic model where the propagation time is independent of the traveled distance. The propagation time is simply determined by a module parameter.
        \item \nedtype{ConstantSpeedPropagation} is a more realistic model where the propagation time is proportional to the traveled distance. The propagation time is independent of the transmitter and receiver movement during both signal transmission and propagation. The propagation speed is determined by a module parameter.
\end{itemize}

The default propagation model is configured as follows:

\inisnippet{PropagationModelConfigurationExample}{Propagation model configuration example}

A more accurate model could take into consideration the transmitter and
receiver movement. This effect becomes especially important for acoustic
communication, because the propagation speed of the signal is much more
comparable to the speed of the transceivers.

\section{Path Loss Models}

As a signal propagates through space its power density decreases. This is
called path loss and it is the combination of many effects such as
free-space loss, refraction, diffraction, reflection, and absorption. There
are several different path loss models in the literature, which differ in
their parameterization and application area.

In INET, a path loss model is an OMNeT++ simple module implementing a
specific path loss algorithm. Its main purpose is to compute the power loss
for a given signal, but it's also capable of estimating the range for a
given loss. The latter is useful, for example, to allow visualizing
communication range. INET contains a number of built-in path loss
algorithms, each comes with its own set of parameters:

\begin{itemize}
        \item \nedtype{FreeSpacePathLoss} models line of sight path loss for air or vacuum.
        \item \nedtype{BreakpointPathLoss} refines it using dual slope model with two separate path loss exponents.
        \item \nedtype{LogNormalShadowing} models path loss for a wide range of environments (e.g. urban areas, and buildings)
        \item \nedtype{TwoRayGroundReflection} models interference between line of sight and single ground reflection.
        \item \nedtype{TwoRayInterference} refines the above for inter-vechicle communication.
        \item \nedtype{RicianFading} is a stochastical model for the anomaly caused by partial cancellation of a signal by itself.
        \item \nedtype{RayleighFading} is a stochastical model for heavily built-up urban environments when there is no dominant propagation along the line of sight.
        \item \nedtype{NakagamiFading} further refines the above two models for cellular systems.
\end{itemize}

The following example replaces the default free-space path loss model with
log normal shadowing:

\inisnippet{PathLossConfigurationExample}{Path loss configuration example}

\section{Obstacle Loss Models}

When the signal propagates through space it also passes through physical
objects present in that space. As the signal penetrates physical objects,
its power decreases when it reflects from surfaces, and also when it’s
absorbed by their material. There are various ways to model this effect,
which differ in the trade-off between accuracy and performance.

In INET, an obstacle loss model is an OMNeT++ simple module. Its main
purpose is to compute the power loss based on the traveled path and the
signal frequency. The obstacle loss models most often use the physical
environment model to determine the set of penetrated physical objects.
INET contains a few built-in obstacle loss models:

\begin{itemize}
        \item \nedtype{IdealObstacleLoss} model determines total or no power loss at all by checking if there is any obstructing physical object along the straight propagation path.
        \item \nedtype{DielectricObstacleLoss} computes the power loss based on the accurate dielectric and reflection loss along the straight path considering the shape, position, orientation, and material of obstructing physical objects.
\end{itemize}

By default, the medium module doesn't contain any obstacle loss model, but
configuring one is very simple:

\inisnippet{ObstacleLossModelConfigurationExample}{Obstacle loss model configuration example}

Statistical obstacle loss models are also possible but currently not provided.

\section{Background Noise Models}

Thermal noise, cosmic background noise, and other random fluctuations of
the electromagnetic field affect the quality of the communication channel.
This kind of noise doesn’t come from a particular source, so it doesn’t
make sense to model its propagation through space. The background noise
model describes instead how it changes over space and time.

In INET, a background noise model is an OMNeT++ simple module. Its main
purpose is to compute the analog representation of the background noise for
a given space-time interval. For example,
\nedtype{IsotropicScalarBackgroundNoise} computes a background noise that is
independent of space-time coordinates, and its scalar power is determined
by a module parameter.

The simplest background noise model can be configured as follows:

\inisnippet{BackgroundNoiseModelConfigurationExample}{Background noise model configuration example}

\section{Analog Models}

The analog signal is a complex physical phenomenon which can be modeled in
many different ways. Choosing the right analog domain signal representation
is the most important factor in the trade-off between accuracy and
performance. The analog model of the transmission medium determines how
signals are represented while being transmitted, propagated, and received.

In INET, an analog model is an OMNeT++ simple module. Its main purpose is
to compute the received signal from the transmitted signal. The analog
model combines the effect of the antenna, path loss, and obstacle loss
models. Transceivers must be configured transmit and receive signals
according to the representation used by the analog model.

The most commonly used analog model, which uses a scalar signal power
representation over a frequency and time interval, can be condigured as
follows:

\inisnippet{AnalogModelConfigurationExample}{Analog model configuration example}

\section{Neighbor Cache}

Transceivers are considered neighbors if successful communication is
possible between them. For wired communication it’s easy to determine
which transceivers are neighbors, because they are connected by wires. In
contrast, in wireless communication determining which transceivers are
neighbors isn’t obvious at all.

In INET, a neighbor cache model is an OMNeT++ simple module which provides
an efficient way of keeping track of the neighbor relationship between
transceivers. Its main purpose is to compute the set of affected receivers
for a given transmission. All built-in models in INET provide a
conservative approximation only, because they update their state
periodically:

\begin{itemize}
        \item \nedtype{NeighborListNeighborCache} maintains a separate neighbor list for each transceiver.
        \item \nedtype{GridNeighborCache} organizes transceivers in a 3D grid with constant cell size.
        \item \nedtype{QuadTreeNeighborCache} organizes transceivers in a 2D quad tree (ignoring the Z axis) with constant node size.
\end{itemize}

\inisnippet{NeighborCacheModelConfigurationExample}{Neighbor cache model configuration example}

%%% Local Variables:
%%% mode: latex
%%% TeX-master: "usman"
%%% End:

\cleardoublepage

\chapter{The Physical Environment}
\label{cha:environment}

\section{Overview}

Wireless networks are heavily affected by the physical environment, and the
requirements for today's ubiquitous wireless communication devices are
increasingly demanding. Cellular networks serve densely populated urban
areas, wireless LANs need to be able to cover large buildings with several
offices, low-power wireless sensors must tolerate noisy industrial
environments, batteries need to remain operational under various external
conditions, and so on.

The propagation of radio signals, the movement of communicating agents,
battery exhaustion, etc., depend on the surrounding physical environment.
For example, signals can be absorbed by objects, can pass through objects,
can be refracted by surfaces, can be reflected from surfaces, or battery
nominal capacity might depend on external temperature. These effects cannot
be ignored in high-fidelity simulations.

In order to help the modeling process, the model of the physical
environment is separated from the rest of the simulation models. The main
goal of the physical environment model is to describe buildings, walls,
vegetation, terrain, weather, and other physical objects and conditions
that might have effects on radio signal propagation, movement, batteries,
etc. This separation makes the model reusable by all other simulation
models that depend on these circumstances.

The following sections provide a brief overview of the physical environment
model.

\section{The Physical Environment Model}

The physical environment is represented in an INET simulation by a
\nedtype{PhysicalEnvironment} module. This module normally has one instance
in the network, and acts as a database that other parts of the simulation
can query at runtime.

\section{Global Physical Properties}

The physical environment model stores the following global
properties:

\begin{itemize}
  \item \textit{space limits}: global bounds for the 3-dimensional space
  \item \textit{temperature}: global parameter for temperature-dependent models
\end{itemize}

Space limits are useful for limiting the propagation and reflection of
radio signals, to constrain movement of communicating agents, and for
detecting incorrectly positioned physical objects.

Temperature can be useful for modeling batteries, as it affects the
maximum capacity, internal resistance, self-discharge and other properties
of real-life electrochemical energy storage devices.

\section{Physical Objects}

The most important aspect of the physical environment is the objects which
are present in it. For example, simulating an indoor Wifi scenario may need
to model walls, floors, ceilings, doors, windows, furniture, and similar
objects, because they all affect signal propagation.

Objects are located in space, and have shapes and materials. The INET
physical layer infrastructure supports basic shapes and homogeneous
materials, which simplifies description and still allows for a reasonable
approximation of reality. Physical objects in INET have the following
properties:

\begin{itemize}
  \item \textit{shape}: describes the 3-dimensional shape of the object,
    independent of its position and orientation
  \item \textit{position}: determines where the object is located in the 3-dimensional space
  \item \textit{orientation}: determines how the object is rotated relative to its
    default orientation
  \item \textit{material}: describes material-specific physical properties
  \item \textit{graphical properties}: provides parameters for better visualization
\end{itemize}

Physical objects in INET are stationary, they cannot change their position
or orientation over time.

Since the shape of the physical objects might be quite diverse, the model
is designed to be extensible with new shapes. Concave shapes are not yet
supported, such shapes can be represented by splitting them up into smaller,
convex parts. The current implementation provides the following shapes:

\begin{itemize}
  \item \textit{sphere}: specified by a radius
  \item \textit{cuboid}: specified by a length, a width, and a height
  \item \textit{prism}: specified by a 2-dimensional polygon base and a height
  \item \textit{polyhedron}: specified by the convex hull of a set of
    3-dimensional vertices
\end{itemize}

\section{Visualization}

The \nedtype{PhysicalEnvironment} module is capable is visualizing the objects on the
user interface. Rendering makes use of the following graphical object properties:

\begin{itemize}
  \item \textit{line width}: affects surface outline
  \item \textit{line color}: affects surface outline
  \item \textit{fill color}: affects surface fill
  \item \textit{opacity}: affects surface outline and fill
  \item \textit{tags}: allows filtering objects on the graphical user interface
\end{itemize}

The projection of 3D objects to the 2D canvas can be parameterized with an
arbitrary view angle. The default view angle is the Z axis (i.e. top view).
The view angle can also be changed during runtime, by changing the
appropriate module parameter.

The projection mechanism can be accessed by other models (e.g. mobility
models) for their own visualizations.

\section{Specifying Physical Objects}

Physical objects are defined for the \nedtype{PhysicalEnvironment} module
in an XML document. The document format allows one to define physical objects
together with their properties, and one can also define shapes and materials
that are shared (i.e. referenced) by several objects.

The following example shows some shapes, materials and objects defined in XML:

\begin{verbatim}
<environment>
  <!-- shapes and materials -->
  <shape id="1" type="sphere" radius="10"/>
  <shape id="2" type="cuboid" size="20 30 40"/>
  <shape id="3" type="prism" height="100"
         points="0 0 100 0 100 100 0 100"/>
  <shape id="4" type="polyhedron"
         points="0 0 0 100 0 0 100 100 0 0 100 0 ..."/>
  <material id="1" resistivity="100"
         relativePermittivity="4.5" relativePermeability="1"/>

  <!-- an object that uses a previously defined shape and material -->
  <object position="min 100 200 0" orientation="45 -30 0"
          shape="1" material="1"
          line-color="0 0 0" fill-color="112 128 144" opacity="0.5"/>

  <!-- an object defined with an in-line shape -->
  <object position="min 100 200 0" orientation="45 -30 0"
          shape="cuboid 20 30 40" material="concrete"
          line-color="0 0 0" fill-color="112 128 144" tags="Building"/>
</environment>
\end{verbatim}

For more details, please refer to the documentation of the
\nedtype{PhysicalEnvironment} module.

\section{Data Structure}

In order to model the physical environment in detail, a scenario might contain
several thousands or even more physical objects. Simulation models might
need to query these objects quite often. For example, when the physical layer
computes obstacle loss for a transmission, it needs to find the obstructing
physical objects for each receiver. This requires computing the intersection
between physical objects and the path traveled by the radio signal.

To speed up the computation of intersections, INET stores physical objects
in a highly efficient data structure, which currently can be one of the
following:

\begin{itemize}
  \item \nedtype{GridObjectCache}: organizes objects into a 3D spatial grid with
    a configurable constant cell size, where cells contain the objects that
    intersect with them
  \item \nedtype{BvhObjectCache}: organizes objects into a 3D tree, where
    leaves contain a configurable number of closely positioned objects.
    (This data structure is similar to quadtree and octree, but is designed for
    storing finite-sized objects.)
\end{itemize}

The physical environment model uses \nedtype{GridObjectCache} by default.


\section{///////////// NEW STUFF://////////////}

\section{Ovewview}

% Why physical environments are needed?
Wireless networks are heavily affected by the physical environment.
Propagation of signals, movement of communicating agents, energy
consumption of devices, all depend on the surrounding physical environment.
For example, signals are absorbed by physical objects and reflected from
their surfaces, or battery capacity might depend on external temperature.
The main purpose of the physical environment model is to describe
buildings, walls, furniture, vegetation, terrain, weather, and other
physical objects and conditions that might have profound effects on the
simulation.

In INET, the physical environment is modeled by the
\nedtype{PhysicalEnvironment} compound module. This module normally has one
instance in the network, and it provides services for other parts of the
simulation. It contains submodules to model physical objects and the ground
as well as a few parameters for the physical properties of the environment.

\section{Physical Objects}
% Why physical objects are needed?

The most important aspect of the physical environment is the objects which
are present in it. For example, simulating an indoor Wifi scenario may need
to model walls, floors, ceilings, doors, windows, furniture, and similar
objects, because they all affect signal propagation. Objects are located in
space, and have shapes and materials. The physical environment model
supports basic shapes and homogeneous materials, which is a simplified
description but still allows for a reasonable approximation of reality.
Physical objects in INET have the following properties:

% What are the properties of physical objects?
\begin{itemize}
        \item \emph{shape} describes the object in 3D independent of its position and orientation.
        \item \emph{position} determines where the object is located in the 3D space.
        \item \emph{orientation} determines how the object is rotated relative to its default orientation.
        \item \emph{material} describes material specific physical properties.
        \item \emph{graphical properties} provide parameters for better visualization.
\end{itemize}

Physical objects in INET are stationary, they cannot change their position
or orientation over time. Since the shape of the physical objects might be
quite diverse, the model is designed to be extensible with new shapes.
INET provides the following shapes:

\begin{itemize}
        \item \emph{sphere} shapes are specified by a radius
        \item \emph{cuboid} shapes are specified by a length, a width, and a height
        \item \emph{prism} shapes are specified by a 2D polygon base and a height
        \item \emph{polyhedron} shapes are specified by the convex hull of a set of 3D vertices
\end{itemize}

The following example shows how to define various physical objects using
the XML syntax supported by the physical environment:

\xmlsnippet{DefiningPhysicalObjectsExample}{Defining physical objects example}

In order to load the above XML file, the following configuration could be
used:

\inisnippet{PhysicalObjectsConfigurationExample}{Physical objects configuration example}

\section{Ground Models}

% Why ground models are needed?
In inter-vehicle simulations the terrain has profound effects on signal
propagation. For example, vehicles on the opposite sides of a mountain
cannot directly communicate with each other.

% What is the purpose of ground models?
A ground model describes the 3D surface of the terrain. Its main purpose is
to compute a position on the surface underneath an particular position.

% What ground models are available?
INET contains the following built-in ground models implemented as
OMNeT++ simple modules:

\begin{itemize}
        \item \nedtype{FlatGround} is a trivial model which provides a flat surface parallel to the XY plane at a certain height.
        \item \nedtype{OsgEarthGround} is a more realistic model (based on \program{osgEarth}) which provides a terrain surface.
\end{itemize}

\section{Geographic Coordinate System Models}

% Why geographic coordinate systems are needed?
In order to run high fidelity simulations, it is often required to embed
the communication network into a real world map. With the new OMNeT++ 5
version, INET already provides support for 3D maps using
\program{osgEarth} for visualization and \program{openstreetmap} as the map
provider.

However, INET carries out all geometric computation internally (including
signal propagation and path loss) in a 3D Euclidean coordinate system. The
discrepancy between the internal playground coordinate system and the usual
geographic coordinate systems must be resolved.

% What is the purpose of geographic coordinate system models?
A geographic coordinate system model maps playground coordinates to
geographic coordinates, and vice versa. Such a model allows positioning
physical objects and describing network node mobility using geographical
coordinates (e.g longitude, latitude, altitude).

% What geographic coordinate system models are provided?
In INET, a geographic coordinate system model is implemented as an OMNeT++
simple module:

\begin{itemize}
        \item \nedtype{SimpleGeographicCoordinateSystem} provides a trivial linear approximation without any external dependency.
        \item \nedtype{OsgGeographicCoordinateSystem} provides an accurate mapping using the external \program{osgEarth} library.
\end{itemize}

% How geographic coordinate system modules are used?
In order to use geographic coordinates in a simulation, a geographic
coordinate system module must be included in the network. The desired
physical environment module and mobility modules must be configured (using
module path parameters) to use the geographic coordinate system module. The
following example also shows how the geographic coordinate system module
can be configured to place the playground at a particular geographic
location and orientation.

\inisnippet{GeographicCoordinateSystemConfigurationExample}{Geographic coordinate system configuration example}

\subsubsection*{Object Cache Models}

% Why object cache models are needed?
If a simulation contains a large number of physical objects, then signal
propagation may become computationally very expensive. The reason is that
the transmission medium model must check each line of sight path between
all transmitter and receiver pairs against all physical objects.

% What is the purpose of object cache models?
An object cache model organizes physical objects into a data structure
which provides efficient geometric queries. Its main purpose is to iterate
all physical objects penetrated by a 3D line segment.

% What object cache models are provided?
In INET, an object cache model is implemented as an OMNeT++ simple module:

\begin{itemize}
        \item \nedtype{GridObjectCache} organizes objects into a fixed cell size 3D spatial grid.
        \item \nedtype{BvhObjectCache} organizes objects into a tree data structure based on recursive 3D volume division.
\end{itemize}


%%% Local Variables:
%%% mode: latex
%%% TeX-master: "usman"
%%% End:


\cleardoublepage

\include{ch-80211}
\cleardoublepage

\include{ch-sensor-macs}
\cleardoublepage

\include{ch-ipv4}
\cleardoublepage

\include{ch-ipv4-config}
\cleardoublepage

\include{ch-ipv6}
\cleardoublepage

\include{ch-transport}
\cleardoublepage

\include{ch-routing}
\cleardoublepage

\include{ch-adhoc-routing}
\cleardoublepage

\include{ch-diffserv}
\cleardoublepage

\include{ch-mpls}
\cleardoublepage

\include{ch-apps}
\cleardoublepage

\chapter{Node Mobility}
\label{cha:mobility}

\section{Overview}

% Why mobility models are needed?
In order to simulate ad-hoc wireless networks, it is important to model the
motion of mobile network nodes. Received signal strength, signal
interference, and channel occupancy depends on the distances between nodes.
The selected mobility models can significantly influence the results of the
simulation (e.g. through packet loss).

% What is the purpose of mobility models?
A mobility model describes movement and orientation over time in a 3D
Euclidean coordinate system. Its main purpose is to determine position,
velocity, acceleration, and similarly angular position, angular velocity,
and angular acceleration as 3D quantities at the current simulation time.

% How are mobility models implemented?
In INET, a mobility model is most often an OMNeT++ simple module
implementing the motion as a C++ algorithm. Although most models have a few
common parameters (e.g. for initial positioning), they always come with
their own set of parameters. Some models support geographic positioning to
ease the configuration of map based scenarios.

\section{Mobility Model Categories}

% How are mobility models categorized?
Mobility models can be divided into trace-based, deterministic, and
stochastic categories. Trace-based mobility models replay recorded motion
as observed in real life. Deterministic and stochastic mobility models use
synthetic mathematical models for describing motion.

Mobility models can also be divided into static, single, and group
categories. Static mobility models provide positioning without any motion
during the simulation. Single mobility models describe the motion of
entities independent of each other. Finally, group mobility models provide
such a motion where group members are dependent on each other.

\section{Built-in Mobility Models}

% What mobility models are provided?
INET contains many mobility models, so the following lists only give a
taste from various categories.

Stationary mobility models:

\begin{itemize}
        \item \nedtype{StationaryMobility} provides deterministic and stochastic stationary positioning.
        \item \nedtype{StaticGridMobility} places several mobility models in a rectangular grid.
        \item \nedtype{StaticConcentricMobility} places several models in a set of concentric circles.
\end{itemize}

Deterministic mobility models:

\begin{itemize}
        \item \nedtype{LinearMobility} moves linearly with a constant speed or constant acceleration.
        \item \nedtype{CircleMobility} moves around a circle parallel to the XY plane with constant speed.
        \item \nedtype{RectangleMobility} moves around a rectangular area parallel to the XY plane with constant speed.
        \item \nedtype{TractorMobility} moves similarly to a tractor on a field with a number of rows.
        \item \nedtype{VehicleMobility} moves similarly to a vehicle along a path especially turning around corners.
        \item \nedtype{TurtleMobility} moves according to an XML script written in a simple yet expressive LOGO-like programming language.
        \item \nedtype{FacingMobility} orients towards the position of another mobility model.
        \item \nedtype{RotatingMobility} rotates around with a constant speed.
\end{itemize}

Trace-based mobility models:

\begin{itemize}
        \item \nedtype{BonnMotionMobility} replays trace files of the \program{BonnMotion} scenario generator.
        \item \nedtype{Ns2MotionMobility} replays files of the CMU’s scenario generator used in \program{Ns2}.
        \item \nedtype{AnsimMobility} replays XML trace files of \program{ANSim} (Ad-Hoc Network Simulation).
\end{itemize}

Stochastic mobility models:

\begin{itemize}
        \item \nedtype{RandomWaypointMobility} moves to random destination with random speed.
        \item \nedtype{GaussMarkovMobility} uses one parameter to vary the degree of randomness from linear to brown motion.
        \item \nedtype{MassMobility} moves similarly to a mass with inertia and momentum.
        \item \nedtype{ChiangMobility} uses a probabilistic transition matrix to change the motion state.
\end{itemize}

\section{Combining Mobility Models}

% How are mobility models combined?
Mobility models can also be combined to form more complex motions without implementing new C++ algorithms.

\begin{itemize}
        \item \nedtype{SuperpositioningMobility} model combines several other mobility models by summing them up. It allows creating group mobility by sharing a mobility model in each group member, separating initial positioning from positioning during the simulation, and separating positioning from orientation.
        \item \nedtype{AttachedMobility} models a mobility that is attached to another one at a given offset. Position, velocity and acceleration are all affected by the respective quantites and also the orientation of the referenced mobility.
\end{itemize}

\section{Using Mobility Models}

% How are mobility models used? A mobility model only determines a
trajectory, but it does so independently of where and how it's used. In
order to actually have an effect on the motion of a network node, a
mobility model is included as a submodule in the compound module of the
network node. By default, a transceiver antenna within a network node uses
the same mobility model, but this is completely optional as far as the
mobility is concerned. For example, a vehicle facing forward while moving
on a road may contain multiple transceiver antennas at different relative
locations with different orientations, these motions cannot be accurately
modeled with a single mobility.



\section{//////// OLD STUFF ///////////}
\section{Overview}

In order to accurately evaluate a protocol for an ad-hoc network,
it is important to use a realistic model for the motion of mobile
hosts. Signal strengths, radio interference and channel occupancy
depends on the distances between nodes. The choice of the mobility
model can significantly influence the results of a simulation
(e.g. data packet delivery ratio, end-to-end delay, average hop count)
as shown in \cite{Camp02asurvey}.

There are two methods for incorporating mobility into simulations:
using traces and synthetic models. Traces contains recorded motion
of the mobile hosts, as observed in real life system. Synthetic models
use mathematical models for describing the behaviour of the mobile hosts.

There are mobility models that represent mobile nodes whose movements
are independent of each other (entity models) and mobility models
that represent mobile nodes whose movements are dependent on each other
(group models). Some of the most frequently used entity models are the
Random Walk Mobility Model, Random Waypoint Mobility Model, Random
Direction Mobility Model, Gauss-Markov Mobility Model, City Section
Mobility Model. The group models include the Column Mobility Model,
Nomadic Community Mobility Model, Pursue Mobility Model,
Reference Point Group Mobility Model.

The INET framework has components for the following trace files:

\begin{itemize}
\item \tbf{Bonn Motion} native file format of the
    \href{http://net.cs.uni-bonn.de/wg/cs/applications/bonnmotion/}{BonnMotion}
    scenario generation tool.
\item \tbf{Ns2} trace file generated by the CMU's scenario generator that used in Ns2.
\item \tbf{ANSim} XML trace file of the ANSim (Ad-Hoc Network Simulation) tool.
\end{itemize}

It is easy to integrate new entity mobility models into the INET framework,
but group mobility is not supported yet. Therefore all the models
shipped with INET are implementations of entitiy models:

\begin{itemize}
\item \tbf{Deterministic Motions} for fixed position nodes and nodes
      moving on a linear, circular, rectangular paths.
\item \tbf{Random Waypoint} model includes pause times between changes
      in destination and speed.
\item \tbf{Gauss-Markov} model uses one tuning parameter to vary the degree
      of randomness in mobility pattern.
\item \tbf{Mass Mobility} models a mass point with inertia and momentum.
\item \tbf{Chiang Mobility} uses a probabilistic transition matrix to change
      the state of motion of the node.
\end{itemize}

\section{Mobility in INET}

In INET mobile nodes have to contain a module implementing the
\nedtype{IMobility} marker interface. This module stores the current
coordinates of the node and is responsible for updating the position
periodically and emitting the \ttt{mobilityStateChanged} signal
when the position changed.

Many mobility models allow the user to define a cubic volume that the node 
can not leave. The volume is configured by setting the \fpar{constraintAreaX}, 
\fpar{constraintAreaY}, \fpar{constraintAreaZ},
\fpar{constraintAreaWidth}, \fpar{constraintAreaHeight} and
\fpar{constraintAreaDepth} parameters.

If the \fpar{initFromDisplayString} parameter, the initial position is taken from
the display string. Otherwise the position can be given as the \fpar{initialX},
\fpar{initialY} and \fpar{initialZ} parameters. If neither of these parameters
are given, a random initial position is choosen within the contraint area.

When the node reaches the boundary of the constraint area, the mobility
component has to prevent the node to exit. Many mobility models offer the 
following policies:

\begin{itemize}
  \item reflect of the wall
  \item reappear at the opposite edge (torus area)
  \item placed at a randomly chosen position of the area
  \item stop the simulation with an error
\end{itemize}

The p[0] and p[1] fields of the display string of the node is
also updated, so if the simulation run is animated, the node is
actually moving on the screen. The current position of the node
can be obtained from the display string.

The radio simulations has a \nedtype{ChannelControl} module that takes case of
establishing communication channels between nodes that are within
communication distance and tearing down the connections when they
move out of range. The \nedtype{ChannelControl} module uses to
\ttt{mobilityStateChanged} signal to determine when the connection
status needs to be updated.

There are two possibilities to implement a new mobility model. The simpler but
limited one is to use \nedtype{TurtleMobility} as the mobility component and to
write a script similar to the turtle graphics of LOGO. The second is to
implement a simple module in C++. In this case the C++ class of the mobility
module should be derived from \cppclass{IMobility} and its NED type should
implement the \nedtype{IMobility} interface.

%\begin{figure}
%\begin{center}
%\includegraphics{figures/mobility_classes}
%\end{center}
%\end{figure}

% FIXME The Z coordinate is often initialized randomly.
%       It would be better (backward compatibility)
%       to initialize them to 0. (contsraintAreaDepth=0 not handled everywhere)


\section{Implemented models}

\subsection{Deterministic movements}

\begin{description}

\item[StationaryMobility] This mobility module does nothing;
it can be used for stationary nodes.

\item[StaticGridMobility] Places all nodes in a rectangular grid.

% TODO it always creates an N x N grid, generalize

\item[LinearMobility] This is a linear mobility model with speed,
angle and acceleration parameters. Angle only changes when the mobile
node hits a wall: then it reflects off the wall at the same angle.

z coordinate is constant
movement is always parallel with X-Y plane

% TODO interpret 'angle' as asimuth and introduce inclination angle (default is 0)
%      to describe movement along arbitrary line segment
% FIXME why different speed and lastSpeed

\item[CircleMobility] Moves the node around a circle parallel to the X-Y plane
with constant speed.
The node bounces from the bounds of the constraint area.
The circle is given by the \fpar{cx}, \fpar{cy} and \fpar{r} parameters,
The initial position determined by the \fpar{startAngle} parameter.
The position of the node is refreshed in \fpar{updateInterval} steps.

\item[RectangleMobility] Moves the node around the constraint area.
configuration: speed, startPos, updateInterval
% should be derived from LineSegmentsMobilityBase?

\item[TractorMobility] Moves a tractor through a field with a certain
amount of rows. The following figure illustrates the movement of the
tractor when the \fpar{rowCount} parameter is 2. The trajectory follows
the segments in $1,2,3,4,5,6,7,8,1,2,3\ldots$ order. The area is configured
by the \fpar{x1}, \fpar{y1}, \fpar{x2}, \fpar{y2} parameters.

% TODO use constraint area instead of new x1,y1,x2,y2 parameters as in RectangleMobility

\begin{center}
\setlength{\unitlength}{0.5mm}
\begin{picture}(80,80)
\put(40,72){$1$} \put(10,70){\vector(1,0){30}} \put(10,70){\line(1,0){60}}
\put(72,55){$2$} \put(70,70){\vector(0,-1){15}} \put(70,70){\line(0,-1){30}}
\put(40,42){$3$} \put(70,40){\vector(-1,0){30}} \put(70,40){\line(-1,0){60}}
\put(5,25){$4$} \put(10,40){\vector(0,-1){15}} \put(10,40){\line(0,-1){30}}
\put(40,12){$5$} \put(10,10){\vector(1,0){30}} \put(10,10){\line(1,0){60}}
\put(72,25){$6$} \put(70,10){\vector(0,1){15}} \put(70,10){\line(0,1){30}}
\put(40, 33){$7$}
\put(5,55){$8$} \put(10,40){\vector(0,1){15}} \put(10,40){\line(0,1){30}}
\put(0,72){$(x_1,y_1)$} \put(65,2){$(x_2,y_2)$}
\end{picture}
\end{center}

\end{description}

\subsection{Random movements}

\begin{description}

\item[RandomWPMobility]

In the Random Waypoint mobility model the nodes move in line segments. For each
line segment, a random destination position (distributed uniformly over the
playground) and a random speed is chosen. You can define a speed as a variate
from which a new value will be drawn for each line segment; it is customary to
specify it as \ttt{uniform(minSpeed, maxSpeed)}. When the node reaches the
target position, it waits for the time \fpar{waitTime} which can also be defined as a
variate. After this time the the algorithm calculates a new random position, etc.

\item[GaussMarkovMobility] The Gauss-Markov model contains a tuning
parameter, that control the randomness in the movement of the node.
Let the magnitude and direction of speed of the node at the $n$th time step be
$s_n$ and $d_n$. The next speed and direction is computed as

$$ s_{n+1} = \alpha s_n + (1 - \alpha) \bar{s} +
             \sqrt{(1-\alpha^2)} s_{x_n} $$

$$ d_{n+1} = \alpha s_n + (1 - \alpha) \bar{d} +
             \sqrt{(1-\alpha^2)} d_{x_n} $$

where $\bar{s}$ and $\bar{d}$ are constants representing the mean value
of speed and direction as $n \to \infty$; and $s_{x_n}$ and $d_{x_n}$
are random variables with Gaussian distribution.

Totally random walk (Brownian motion) is obtained by setting $\alpha=0$,
while $\alpha=1$ results a linear motion.

To ensure that the node does not remain at the boundary of the constraint
area for a long time, the mean value of the direction ($\bar{d}$) modified
as the node enters the margin area. For example at the right edge of the
area it is set to 180 degrees, so the new direction is away from the edge.

% FIXME the GaussMarkovMobility module has only one variance parameter.
%       it should have separate speed and direction parameters

\item[MassMobility]

This is a random mobility model for a mobile host with
a mass. It is the one used in \cite{Perkins99optimizedsmooth}.

\begin{quote}
"An MH moves within the room according to the following pattern. It moves
along a straight line for a certain period of time before it makes a turn.
This moving period is a random number, normally distributed with average of
5 seconds and standard deviation of 0.1 second. When it makes a turn, the
new direction (angle) in which it will move is a normally distributed
random number with average equal to the previous direction and standard
deviation of 30 degrees. Its speed is also a normally distributed random
number, with a controlled average, ranging from 0.1 to 0.45 (unit/sec), and
standard deviation of 0.01 (unit/sec). A new such random number is picked
as its speed when it makes a turn. This pattern of mobility is intended to
model node movement during which the nodes have momentum, and thus do not
start, stop, or turn abruptly. When it hits a wall, it reflects off the
wall at the same angle; in our simulated world, there is little other
choice."
\end{quote}

This implementation can be parameterized a bit more, via the
\fpar{changeInterval}, \fpar{changeAngleBy} and \fpar{changeSpeedBy} parameters.
The parameters described above correspond to the following settings:

\begin{inifile}
changeInterval = normal(5, 0.1)
changeAngleBy = normal(0, 30)
speed = normal(avgSpeed, 0.01)
\end{inifile}

\item[ChiangMobility] Chiang's random walk movement model
(\cite{Chiang98wirelessnetwork}).

In this model, the state of the mobile node in each direction (x and y) can be:

\begin{itemize}
  \item 0: the node stays in its current position
  \item 1: the node moves forward
  \item 2: the node moves backward
\end{itemize}

The $(i,j)$ element of the state transition matrix determines the
probability that the state changes from $i$ to $j$:

$$ \left(
\begin{array}{ccc}
  0 & 0.5 & 0.5 \\
  0.3 & 0.7 & 0 \\
  0.3 & 0 & 0.7
\end{array}
\right) $$

The \nedtype{ChiangMobility} module supports the following parameters:
\begin{itemize}
  \item \fpar{updateInterval} position update interval
  \item \fpar{stateTransitionInterval} state update interval
  \item \fpar{speed}: the speed of the node
\end{itemize}


% FIXME last line of setTargetPosition() contains a sign error, should be
%              targetPosition = lastPosition + lastSpeed * stateTransitionUpdateInterval;
% FIXME when reflecting at the boundary, state variables should be reflected too

\item[ConstSpeedMobility]

\nedtype{ConstSpeedMobility} does not use one of the standard mobility
approaches. The user can define a velocity for each Host and an update interval. If
the velocity is greater than zero (i.e. the Host is not stationary) the
\nedtype{ConstSpeedMobility} module calculates a random target position for the
Host. Depending to the update interval and the velocity it calculates the number of
steps to reach the destination and the step-size. Every update interval
\nedtype{ConstSpeedMobility} calculates the new position on its way to the
target position and updates the display. Once the target position is reached
\nedtype{ConstSpeedMobility} calculates a new target position.

This component has been taken over from Mobility Framework 1.0a5.

% FIXME this is a special case of RandomWPMobility, remove

\end{description}

\subsection{Replaying trace files}

\begin{description}

\item[BonnMotionMobility] Uses the native file format of
\href{http://www.cs.uni-bonn.de/IV/BonnMotion/}{BonnMotion}.

The file is a plain text file, where every line describes the motion
of one host. A line consists of one or more (t, x, y) triplets of real
numbers, like:

\begin{verbatim}
t1 x1 y1 t2 x2 y2 t3 x3 y3 t4 x4 y4 ...
\end{verbatim}

The meaning is that the given node gets to $(xk,yk)$ at $tk$. There's no
separate notation for wait, so x and y coordinates will be repeated there.

\item[Ns2Mobility] Nodes are moving according to the trace files used
in NS2.
The trace file has this format:

\begin{verbatim}
# '#' starts a comment, ends at the end of line
$node_(<id>) set X_ <x> # sets x coordinate of the node identified by <id>
$node_(<id>) set Y_ <y> # sets y coordinate of the node identified by <id>
$node_(<id>) set Z_ <z> # sets z coordinate (ignored)
$ns at $time "$node_(<id>) setdest <x> <y> <speed>" # at $time start moving
towards <x>,<y> with <speed>
\end{verbatim}

The \nedtype{Ns2MotionMobility} module has the following parameters:

\begin{itemize}
  \item \fpar{traceFile} the Ns2 trace file
  \item \fpar{nodeId} node identifier in the trace file; -1 gets substituted by
  parent module's index
  \item \fpar{scrollX},\fpar{scrollY} user specified translation of the
  coordinates
\end{itemize}

% TODO cleaning the code (e.g. duplicated bounds check in setTargetPosition())
% TODO implement cached file access as in BonnMotionMobility

\item[ANSimMobility] reads trace files of the \href{http://www.ansim.info}{ANSim} Tool.

The nodes are moving along linear segments described by an XML trace file
conforming to this DTD:

\begin{verbatim}
<!ELEMENT mobility (position_change*)>
<!ELEMENT position_change (node_id, start_time, end_time, destination)>
<!ELEMENT node_id (#PCDATA)>
<!ELEMENT start_time (#PCDATA)>
<!ELEMENT end_time (#PCDATA)>
<!ELEMENT destination (xpos, ypos)>
<!ELEMENT xpos (#PCDATA)>
<!ELEMENT ypos (#PCDATA)>
\end{verbatim}

Parameters of the module:

\begin{itemize}
  \item \fpar{ansimTrace} the trace file
  \item \fpar{nodeId} the \verb!node_id! of this node, -1 gets substituted to
  parent module's index
\end{itemize}

\begin{note}
The \nedtype{AnsimMobility} module process only the \ttt{position\_{}change}
elements and it ignores the \ttt{start\_{}time} attribute. It starts the move
on the next segment immediately.
\end{note}


\end{description}




\section{Mobility scripts}

The \nedtype{TurtleMobility} module can be parametrized by a script file
containing LOGO-style movement commands in XML format.

The module has these parameters:

\begin{itemize}
\item \ttt{updateInterval} time interval to update the hosts position
\item \ttt{constraintAreaX}, \ttt{constraintAreaY}, \ttt{constraintAreaWidth},
      \ttt{constraintAreaHeight}: constraint area that the node can not leave
\item \ttt{turtleScript} XML file describing the movements
\end{itemize}

The content of the XML file should conform to the following DTD (can be
found as \ffilename{TurtleMobility.dtd} in the source tree):

\begin{verbatim}
<!ELEMENT movements (movement)*>

<!ELEMENT movement (repeat|set|forward|turn|wait|moveto|moveby)*>
<!ATTLIST movement id NMTOKEN #IMPLIED>

<!ELEMENT repeat (repeat|set|forward|turn|wait|moveto|moveby)*>
<!ATTLIST repeat n CDATA #IMPLIED>

<!ELEMENT set EMPTY>
<!ATTLIST set x     CDATA #IMPLIED
              y     CDATA #IMPLIED
              speed CDATA #IMPLIED
              angle CDATA #IMPLIED
              borderPolicy (reflect|wrap|placerandomly|error) #IMPLIED>

<!ELEMENT forward EMPTY>
<!ATTLIST forward d CDATA #IMPLIED
                  t CDATA #IMPLIED>

<!ELEMENT turn EMPTY>
<!ATTLIST turn angle CDATA #REQUIRED>

<!ELEMENT wait EMPTY>
<!ATTLIST wait t CDATA #REQUIRED>

<!ELEMENT moveto EMPTY>
<!ATTLIST moveto x CDATA #IMPLIED
                 y CDATA #IMPLIED
                 t CDATA #IMPLIED>

<!ELEMENT moveby EMPTY>
<!ATTLIST moveby x CDATA #IMPLIED
                 y CDATA #IMPLIED
                 t CDATA #IMPLIED>
\end{verbatim}

The file contains \ttt{movement} elements, each describing a trajectory.
The \ttt{id} attribute of the \ttt{movement} element can be used to
refer the movement from the ini file using the syntax:

\begin{inifile}
**.mobility.turtleScript = xmldoc("turtle.xml", "movements//movement[@id='1']")
\end{inifile}

The motion of the node is composed of uniform linear segments.
The state of motion is described by the following variables:
\begin{itemize}
\item \ttt{position}: $(x,y)$ coordinate of the current location of the node
\item \ttt{speed}, \ttt{angle}: magnitude and direction of the node's velocity
\item \ttt{targetPos}: target position of the current line segment. If given
                       the \ttt{speed} and \ttt{angle} is not used
\item \ttt{targetTime} the end time of the current linear motion
\item \ttt{borderPolicy}: one of
    \begin{itemize}
      \item \ttt{reflect} the node reflects at the boundary,
      \item \ttt{wrap} the node appears at the other side of the area,
      \item \ttt{placerandomly} the node placed at a random position of the area,
      \item \ttt{error} signals an error when the node reaches the boundary
    \end{itemize}
\end{itemize}

The \ttt{movement} elements may contain the the following commands:

\begin{itemize}
\item \ttt{repeat(n)} repeats its content n times, or indefinetly if the n attribute
              is omitted.
\item \ttt{set(x,y,speed,angle,borderPolicy)} modifies the state of the node.
\item \ttt{forward(d,t)} moves the node for $t$ time or to the $d$ distance
with the current speed. If both $d$ and $t$ is given, then the current
speed is ignored.
\item \ttt{turn(angle)} increase the angle of the node by $angle$ degrees.
\item \ttt{moveto(x,y,t)} moves to point $(x,y)$ in the given time. If
$t$ is not specified, it is computed from the current speed.
\item \ttt{moveby(x,y,t)} moves by offset $(x,y)$ in the given time. If
$t$ is not specified, it is computed from the current speed.
\item \ttt{wait(t)} waits for the specified amount of time.
\end{itemize}

Attribute values must be given without physical units, distances are assumed
to be given as meters, time intervals in seconds and speeds in meter per seconds.
Attibutes can contain expressions that are evaluated each time the
command is executed. The limits of the constraint area can be
referenced as \verb!$MINX!, \verb!$MAXX!, \verb!$MINY!, and \verb!$MAXY!.
Random number distibutions generate a new random number when evaluated,
so the script can describe random as well as deterministic scenarios.

To illustrate the usage of the module, we show how some mobility
models can be implemented as scripts:

\begin{itemize}
\item RectangleMobility:

\begin{verbatim}
    <movement>
        <set x="$MINX" y="$MINY" angle="0" speed="10"/>
        <repeat>
            <repeat n="2">
                <forward d="$MAXX-$MINX"/>
                <turn angle="90"/>
                <forward d="$MAXY-$MINY"/>
                <turn angle="90"/>
            </repeat>
        </repeat>
    </movement>
\end{verbatim}

\item Random Waypoint:

\begin{verbatim}
    <movement>
        <repeat>
            <set speed="uniform(20,60)"/>
            <moveto x="uniform($MINX,$MAXX)" y="uniform($MINY,$MAXY)"/>
            <wait t="uniform(5,10)">
        </repeat>
    </movement>
\end{verbatim}

\item MassMobility:

\begin{verbatim}
    <movement>
        <repeat>
            <set speed="uniform(10,20)"/>
            <turn angle="uniform(-30,30)"/>
            <forward t="uniform(0.1,1)"/>
        </repeat>
    </movement>
\end{verbatim}

\end{itemize}


%%% Local Variables:
%%% mode: latex
%%% TeX-master: "usman"
%%% End:



\cleardoublepage

\chapter{Modeling Power Consumption}
\label{cha:power}

\section{Overview}

Modeling power consumption becomes more and more important with the increasing
number of embedded devices and the upcoming Internet of Things. Mobile personal
medical devices, large scale wireless environment monitoring devices, electric
vehicles, solar panels, low-power wireless sensors, etc. require paying special
attention to power consumption. The high fidelity simulation of power
consumption allows designing power sensitive routing protocols, MAC protocols,
physical layers, etc. which in turn results in more energy efficient devices.

In order to help the modeling process the power model is separated from other
simulation models. This separation makes the model extensible and it also allows
easy experimentation with alternative implementations. In a nutshell the power
model consists of the following components:

\begin{itemize}
  \item energy consumption models
  \item energy generation models
  \item temporary energy storage models
\end{itemize}

The power model elements fall into two categories, abbreviated with \ttt{Ep}
and \ttt{Cc} as part of their names: 

\begin{itemize}
  \item \ttt{Ep} models are simpler, and deal with energy and power quantities.
  \item \ttt{Cc} models are more realistic, and deal with charge, current, and voltage quantities.
\end{itemize}

The following sections provide a brief overview of the power model.

\section{Energy Consumer Models}

Energy consumer models describe the energy consumption of devices over time. 
For example, a transceiver consumes energy when it transmits or
receives signals, a CPU consumes energy when the network protocol routes
packets, and a display consumes energy when it's turned on.

In INET, an energy consumer model is an OMNeT++ simple module that implements
the energy consumption of software processes or hardware devices over time.
Its main purpose is to provide the power or current consumption for the
current simulation time. Most often energy consumers are included as
submodules in the compound module of the hardware devices or software
components.

INET provides only a few built-in energy consumer models:

\begin{itemize}
        \item \nedtype{AlternatingEpEnergyGenerator} is a trivial energy/power based statistical energy consumer model example.
        \item \nedtype{StateBasedEpEnergyConsumer} is a transceiver energy consumer model based on the radio mode and transmission/reception states.
\end{itemize}

In order to simulate power consumption in a wireless network, the energy
consumer model type must be configured for the transceivers. The following
example demonstrates how to configure the power consumption parameters for
a transceiver energy consumer model:

\inisnippet{EnergyConsumerConfigurationExample}{Energy consumer configuration example}


\section{Energy Generator Models}

Energy generator models describe the energy generation of devices over time. 
A solar panel, for example, produces energy based on time, the panel's 
position on the globe, its orientation towards the sun and the actual weather 
conditions. Energy generators connect to an energy storage that absorbs the 
generated energy. 

In INET, an energy generator model is an OMNeT++ simple module implementing
the energy generation of a hardware device using a physical phenomena over
time. Its main purpose is to provide the power or current generation for
the current simulation time. Most often energy generation models are
included as submodules in network nodes.

INET provides only one trivial energy/power based statistical energy
generator model called \nedtype{AlternatingEpEnergyGenerator}. The following
example shows how to configure its power generation parameters:

\inisnippet{EnergyGeneratorConfigurationExample}{Energy generator configuration example}

\section{Energy Storage Models}

Electronic devices which are not connected to external power source must contain 
some component to store energy. For example, an electrochemical battery in a
mobile phone provides energy for its display, its CPU, and its communication
devices. It might also absorb energy produced by a solar installed on its
display, or by a portable charger plugged into the wall socket.

In INET, an energy storage model is an OMNeT++ simple module which models
the physical phenomena that is used to store energy produced by generators
and provide energy for consumers. Its main purpose is to compute the amount
of available energy or charge at the current simulation time. It maintains
a set of connected energy consumers and energy generators, their respective
total power consumption and total power generation.

INET contains a few built-in energy storage models:

\begin{itemize}
        \item \nedtype{IdealEpEnergyStorage} is an idealistic model with infinite energy capacity and infinite power flow.
        \item \nedtype{SimpleEpEnergyStorage} is a non-trivial model integrating the difference between the total consumed power and the total generated power over time.
        \item \nedtype{SimpleCcBattery} is a more realistic charge/current based battery model using a charge independent ideal voltage source and internal resistance.
\end{itemize}

The following example shows how to configure a simple energy storage model:

\inisnippet{EnergyStorageConfigurationExample}{Energy storage configuration example}

\section{Energy Management Models}

% Why energy management models are needed?
% The energy management models monitors an energy storage, estimates its state, and controls the consumers and generators to protect the energy storage from operating outside its safe operating area.

\nedtype{SimpleEpEnergyManagement}

\inisnippet{EnergyManagementConfigurationExample}{Energy management configuration example}



%%% Local Variables:
%%% mode: latex
%%% TeX-master: "usman"
%%% End:


\cleardoublepage

\include{ch-visualization}
\cleardoublepage

\include{ch-lifecycle}
\cleardoublepage

\chapter{Scenario Management}
\label{cha:scenario-management}

TODO explain how INET supports custom simulation scenarios: creating new 
network nodes, shutting down network nodes, etc.

%%% Local Variables:
%%% mode: latex
%%% TeX-master: "usman"
%%% End:

\cleardoublepage

\include{ch-emulation}
\cleardoublepage

\include{ch-authors-guide}
\cleardoublepage

\include{ch-history}
\cleardoublepage

\bibliographystyle{alpha}
\bibliography{inet-users-guide}


%% no need for the following since 'tocbibind' package
%% \phantomsection
%% \addcontentsline{toc}{chapter}{\indexname}
\printindex

\end{document}

%%% Local Variables:
%%% mode: latex
%%% TeX-master: t
%%% End:
